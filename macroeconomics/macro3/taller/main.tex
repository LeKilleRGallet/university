\documentclass[11pt]{article}
\usepackage{UF_FRED_paper_style}
\usepackage{lipsum}
\onehalfspacing

\setlength{\droptitle}{-5em} %% Don't touch

\title{Taller I monitoria macroeconomia III
}

\author{Augusto Rico\\
    \href{mailto:arico@unal.edu.co}{\texttt{arico@unal.edu.co}}
}

\date{\today}

\begin{document}

{\setstretch{.8} %% Don't touch
\maketitle


% %%%%%%%%%%%%%%%%%%%%%%%%%%%%%%%%%%%%%%%%%%%%%%%%%%%%%%%%%%
% %%%%%%%%%%%%%%%%%%%%%%%%%%%%%%%%%%%%%%%%%%%%%%%%%%%%%%%%%%
% BODY OF THE DOCUMENT
% %%%%%%%%%%%%%%%%%%%%%%%%%%%%%%%%%%%%%%%%%%%%%%%%%%%%%%%%%%
% %%%%%%%%%%%%%%%%%%%%%%%%%%%%%%%%%%%%%%%%%%%%%%%%%%%%%%%%%%

\section{Punto 1}
% --------------------
\begin{flushleft}    Para el modelo \textit{Harrod-Domar} la relación capital-trabajo es fundamental, ya que al ser un modelo de corte keynesiana en el corto plazo se busca 
    el pleno empleo, hecho que únicamente se puede dar si y solo si la tasa de crecimiento del capital coincide con la tasa de crecimiento, la fuerza laboral, 
    no obstante es complejo que se genere esta coincidencia, puesto que no existen mecanismos que puedan obligar la coincidencia en el mercado, ya que este se 
    encuentra en constante incertidumbre, lo que condiciona las expectativas de los inversores, quienes dependiendo su optimismo pueden determinar aumentar o
    no la cantidad de capital instalado por sus decisiones de ahorro, y es que esto se puede entender como un modelo de coeficientes fijos, puesto que si $K>L$ o $K<L$
    se tendrá que habrá una subutilización por parte del coeficiente que es mayor en cada caso, pudiendo generar inflación o desempleo respectivamente, mientras que si $K=L$ se entiende una perfecta
    Utilización de los factores obteniendo el pleno empleo.
\end{flushleft}

% --------------------
\section{Punto 2}
% --------------------
\begin{flushleft}
    Una identidad contable es una igualdad que es verdad siempre para cualesquiera que sean los valores, por ende sin ningún tipo de causalidad implícita en ella, 
    por lo que una identidad contable no se puede considerar una teoría, ya que están no son ciertas siempre ni para todos los casos, sino que, en cambio, se deben considerar como una afirmación siempre verdadera 
    como puede ser el cálculo del PIB, que no tienen ningún ápice de causalidad y únicamente sirve para obtener una suma de partes.
    \\~\\
    Por lo anterior, \citet{shaikh_1974} considera que la "\textit{cobb-douglas}" no es ninguna función de producción, ya que como él expone carece de cualquier causalidad, tanto qué datos completamente 
    falsos le arrojan que son perfectos y encuadran perfectamente con la "\textit{teoría}" detrás de esta identidad contable, hecho que en una función de producción no podrían ser posibles, por el hecho de que en términos de \citet{popper_1959}
    Toda teoría debería poder ser falsable, soportando todos los intentos de refutación de la misma y de este modo poder separar la ciencia de la \textit{pseudo}-ciencia, hecho que no logro la \textit{cobb-douglas}.
\end{flushleft}

\section{Punto 3}
% --------------------
\begin{flushleft}
    Desacuerdo a \citet{friedman_1968} la tasa natural de desempleo es el punto en el largo plazo donde se intercepta la oferta de empleo
    y su demanda en un mercado des-regularizado, y por ende desde la visión neoclásica la tasa de desempleo que debe existir en la economía, dado que no 
    existen deformaciones en la economía por parte de la demanda.
    \\~\\
    Esta visión, por ende, cree entonces que la única manera que existe para modificar la tasa de desempleo es por un cambio en la oferta de la economía, 
    ya que todo estimulo al empleo por parte de la demanda es una perturbación al mercado, inocuo y perjudicial en el largo plazo.
    \\~\\
    \citet{kalecki_1980} considera que esto es parcialmente cierto, puesto que aunque si es cierto para las economías subdesarrolladas con falta de capital 
    que la oferta de este mismo es la determinante del desempleo y la demanda agregada poco va a lograr aportar a mejorar esto, no sería igual para las economías
    desarrolladas ya que estas ya se encuentran en el culmen de su capacidad de capital y lo que necesita es que la población logre consumir para lograr una reducción del desempleo,
    por lo que para disminuir el desempleo se deben hacer políticas por parte de la demanda.
\end{flushleft}

\section{Punto 4}
% --------------------
\begin{flushleft}
    Existen 3 tipos de progreso técnico altamente relevantes, el de Harrod, el de Solow y el de Hicks, donde el primero considera
    la existencia de un aumento en la productividad marginal de trabajo sin cambio en la proporción capital-trabajo, por el contrario, el de Solow considera un aumento en la productividad
    marginal del capital manteniendo la relación capital-trabajo constante y el de Hicks, el cual es el utilizado en el modelo de crecimiento de Solow, 
    no es más que una combinación de las anteriores siempre y cuando la función sea homogénea de grado uno por lo que se tiene un aumento en el \textit{output}
    sin un cambio en las cantidades de capital o trabajo que se representa matemáticamente como

    \begin{equation}
        \underbrace{Y=f(K,L)}_{\text{sin progreso técnico}} \to Y=f(\underbrace{\alpha K}_{\text{Solow}},\underbrace{\alpha L}_{\text{Harrod}}) \to Y = \overbrace{\underbrace{\alpha f(K,L)}_{\text{Hicks}}}^{f(\lambda x) = \lambda^1 f(x)}
    \end{equation}

    y es que dadas las propiedades matemáticas que ofrece la neutralidad de Hicks, propiedades que aportan para obtener grandiosos resultados estadísticos aunque
    banales económicamente se comprende por qué Solow la utilizo para su modelo, ya que se logra una sobresimplificacion de problema y resultados verosímiles que apoyan para validar \textit{pseudo}-científicamente,
    Razón por la que debe ser gran candidata a ser utilizada.
\end{flushleft}

\section{Punto 5}
% --------------------
\begin{flushleft}
    Los debates del capital iniciaron con \citet{robinson_1953} preguntándose \textit{¿en qué unidades se mide el capital?}, ya que la discusión yace sobre la concepción \textit{tradicional} del capital
    que impide la homogeneidad de bienes tan heterogéneos como los \textit{bienes de capital} y mucho menos a homogeneizarlos en unidades monetarias a través del tiempo para poder explicar una redistribución constante en el tiempo
    cuando las unidades monetarias en sí mismas explican esa tasa de interés implícita en el tiempo, por lo que concluye que únicamente puede ser posible esta homogeneización del capital en un mundo estático.
    \\~\\
    Por parte de la escuela neoclásica la respuesta fue bastante vaga y representada principalmente por \citet{samuelson_1962} quien como objetivo tenía reducir la crítica a casos particulares y poco relevante,
    que aunque aceptando la critica como válida en ciertos casos, no en el agregado de la economía por lo que a su consideración esto no afectaría ni la teoría neoclásica en general, ni mucho menos las funciones de producción en particular.
    Lo que fue el pensamiento común de la Ortodoxia, ya que aceptar a cabalidad las críticas del capital hubieran implicado una reformulación de la escuela dominante, algo inconcebible para ellos, puesto que
    en ese caso, por ejemplo, el modeo solow-swan seria inválido en el tiempo para poder explicar la redistribución de forma \textit{"técnica"} sin recurrir a planteamientos clásicos de lucha distributiva.\citep{lazzarini2013}\end{flushleft}

\section{Punto 7}
% --------------------
\begin{flushleft}
    Para lograr el pleno empleo de los factores productivos en el modeo Harrod-Domar es necesario que la cantidad de ambos factores sea equivalente, ya que el aumento en la cantidad de un factor
    manteniendo invariante el otro factor no implica un aumento en el \textit{output}, sino que mantiene constante también el output, lo cual también puede considerarse una perdida de producción, 
    por el hecho de que si esa variación en un factor es mejor redistribuida entre ambos factores si se logra un aumento en el \textit{output}, lo que se puede ver con la siguiente función $\min\{\bar{K},\bar{L}\}=\bar{Y}$, 
    si aumentamos una cantidad $\varepsilon \in \mathbb{R}^+$ a uno de los dos factores manteniendo el otro factor inalterado obtendremos que $\min\{\bar{K}+\varepsilon,\bar{L}\}=\bar{Y}$ donde podemos observa que la producción se mantuvo inalterada,
    no obstante, si en lugar de aumentar únicamente un factor, aprovechamos las características cuasiconcavas de la función de coeficientes fijos de \textit{Leontief} vamos a tener que
    $\min\left\{\bar{K}+\frac{\varepsilon}{2},\bar{L}+\frac{\varepsilon}{2}\right\}=\bar{Y}+\frac{\varepsilon}{2}$ por lo que si vamos a lograr en este caso a diferencia del anterior un pleno
    empleo de los factores productivos obteniendo además que $\bar{Y}+\frac{\varepsilon}{2}>\bar{Y}$
\end{flushleft}
\newpage

% %%%%%%%%%%%%%%%%%%%%%%%%%%%%%%%%%%%%%%%%%%%%%%%%%%%%%%%%%%
% %%%%%%%%%%%%%%%%%%%%%%%%%%%%%%%%%%%%%%%%%%%%%%%%%%%%%%%%%%
% REFERENCES SECTION
% %%%%%%%%%%%%%%%%%%%%%%%%%%%%%%%%%%%%%%%%%%%%%%%%%%%%%%%%%%
% %%%%%%%%%%%%%%%%%%%%%%%%%%%%%%%%%%%%%%%%%%%%%%%%%%%%%%%%%%
\medskip

\bibliography{references.bib} 

\newpage

\end{document}