\documentclass[11pt,letterpaper]{article}
\usepackage{amsmath}
\usepackage[utf8]{inputenc}
\usepackage{graphicx}
\usepackage{wrapfig}

\title{\textbf{Parcial 1 Macroeconomia III}}
\author{Augusto Rico}
\date{\today}

\graphicspath{{Fig/}}

\newcommand*{\captionsource}[2]{%
    \textbf{\\Tomado:} #2%
  }%


\begin{document}
\maketitle

\section{Ley de Say vs demanda efectiva}
\begin{flushleft}
    Say (1803) plantea que en el agregado de la economia la oferta es lo que determina la demanda, esto que puede ser valido en una economia real, 
    segun Keynes (1936) esto es algo que no se puede afirmar para una economia monetaria donde el dinero no es neutral, ya que no se demanda unicamente para realizar otros intercambios.\\
    ~\\
    La principal diferencia entre estas teorias no es mas que la concepcion del dinero, para Say el dinero no es mas que un medio de intercambio, 
    por lo que los agentes no lo demandan por si mismo y al poseerlo desean desesperadamente gastarlo. Por otro lado, Keynes considera que esta neutralidad del dinero no concuerda con la realidad, 
    ya que segun Keynes(1936) los agentes tienen una prefrencia por la liquidez por lo que no se demanda totalmente la produccion debido a que parte del ingreso es finalmente destinado a la demanda del dinero, porque este es util para los individuos mas alla de un metodo de intercambio, como podria ser un costo de oportunidad.\\
    ~\\
    

\end{flushleft}
    


\end{document}