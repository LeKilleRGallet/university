\documentclass[11pt]{article}
\usepackage{UF_FRED_paper_style}
\usepackage{lipsum}
\onehalfspacing

\setlength{\droptitle}{-5em} %% Don't touch

\title{Parcial II Macroeconomia III
}

\author{Augusto Rico\\
    \href{mailto:arico@unal.edu.co}{\texttt{arico@unal.edu.co}}
    }

\date{\today}

\begin{document}

\maketitle

% %%%%%%%%%%%%%%%%%%%%%%%%%%%%%%%%%%%%%%%%%%%%%%%%%%%%%%%%%%
% %%%%%%%%%%%%%%%%%%%%%%%%%%%%%%%%%%%%%%%%%%%%%%%%%%%%%%%%%%
% BODY OF THE DOCUMENT
% %%%%%%%%%%%%%%%%%%%%%%%%%%%%%%%%%%%%%%%%%%%%%%%%%%%%%%%%%%
% %%%%%%%%%%%%%%%%%%%%%%%%%%%%%%%%%%%%%%%%%%%%%%%%%%%%%%%%%%

% --------------------
\section{Punto 1}
% --------------------

\begin{flushleft}
    ~\\
\end{flushleft}

\newpage

\section{Punto 2}
% --------------------

\epigraph{ \itshape Before ever he does ask, he has become a professor, and so sloppy habits of thought are handed on from one generation to the next.}
            {--- Joan Robinson }

\begin{flushleft}
    Desde sus inicios las funciones de produccion en general pero la PTF en particular han sido objeto de debate y controversia. En especial por parte de economistas heterodoxos, 
    quienes han criticado su supuesta falta de consistencia teorica e incluso su falta de validez empirica.
    La primera critica conocida como \textit{controversias del capital entre las dos Cambridge} iniciada por \citet{robinson_1953}. 
    Y la segunda conocida como \textit{la tirania de la identidad contable} hecha por \citet{shaikh_1974}.

    \subsection{Controversias del capital entre las dos Cambridge}
    \citet{robinson_1953} Inicia esta critica preguntandose \textit{en que unidad se mide el Capital},
    que aunque parezca ambigua y laxa, realmente es una pregunta demoledora, ya que 
    la escuela neoclasica no es realmente capaz de dar una respuesta, y es que, partiendo de que el capital se compone tambien de trabajo
    se tendria una funcion de produccion donde unicamente interviene el trabajo, que aunque pueda ser coherente en el corto plazo,
    no lo es fuera de este periodo, ya que al abandonar el corto plazo debemos determinar inicialmente si evaluar el capital en termino
    de sus costos pasados o en cambio de sus rendimientos futuros \textbf{esperados}.
    \\~\\
    El problema radica en que para determinar los rendimientos futuros se debe tener 
    una tasa de interes dada, pero teniendo en cuenta que el principal proposito de la funcion de produccion 
    es mostrar como se distribuyen los ingresos entre trabajadores y capitalistas\footnote{la distribucion de los capitalistas se puede entender como la tasa de interes} de una forma tecnica.
    descartando esto, nos quedaria por evaluar la posiblidad de determinar esto mediante los costos pasados,
    generando el problema de que al agrupar el capital como trabajo nuevamente, y no es posible equiparar las unidades trabajadas en el tiempo por lo que se mantendra inconclusa la pregunta.
    \\~\\
    Por todo esto y adicional a que las expectativas no tienen porque reflejarse en la realidad \textit{ex-post},
    se genera una ambiguedad metodologia llegando a creer que existe posiblidad de moverse en el tiempo hacia el pasado.
    lo que nos debe llevar a la conclusion que no puede existir una teoria del valor valida distinta a la del corto plazo.
    \\~\\
    
    \subsection{La tirania de la identidad contable}
    \citet{shaikh_1974} plantea que la funcion \textit{cobb-douglas} no es ninguna funcion de produccion, 
    sino en cambio una forma algebraica para justificar la teoria neoclasica de la distribucion como de lugar,
    hecho que demuestra facilmente transformando una relacion algebraica sin causalidad en una funcion altamente
    semejante a la forma funcional de la \textit{cobb-douglas}, por lo que unicamente quedaria
    una funcion que explica distribucion, no obstante como las participaciones se consideran
    constantes en el tiempo y es tomado de forma empirica siquiera se puede concluir que esto sea cierto.
    \\~\\
    utilizando lo anterior, Shaikh muestra que todo esto es extrapolable a la funcion de produccion de Sollow,
    llegando de la misma manera algebraica a la funcion subyacente de Solow, exponiendo que todas las conclusiones
    de crecimiento de Solow no son mas que incorrectas conclusiones algebraidas debenidas de una floja forma de pensamiento sin ninguna fuerza teorica.
    \\~\\
    Dado que la \textit{"funcion"} de Solow no es mas que una ley algebraica, procede a mostar la invalidez empirica de la misma utilizando datos aleatorios
    que terminan derivando en las mismas conclusiones distributivas. exponiendo que estos resultados empiricos no tienen relacion alguna con 
    una condicion de produccion sino en cambio en una mera relacion matematica que siempre va a ser satisfecha.
    
    \subsection{Conclusiones}

    


\end{flushleft}

\newpage

\section{Punto 3}
% --------------------

\begin{flushleft}
    ~\\
\end{flushleft}

\newpage

% %%%%%%%%%%%%%%%%%%%%%%%%%%%%%%%%%%%%%%%%%%%%%%%%%%%%%%%%%%
% %%%%%%%%%%%%%%%%%%%%%%%%%%%%%%%%%%%%%%%%%%%%%%%%%%%%%%%%%%
% REFERENCES SECTION
% %%%%%%%%%%%%%%%%%%%%%%%%%%%%%%%%%%%%%%%%%%%%%%%%%%%%%%%%%%
% %%%%%%%%%%%%%%%%%%%%%%%%%%%%%%%%%%%%%%%%%%%%%%%%%%%%%%%%%%
\medskip

\bibliography{references.bib} 

\newpage

\end{document}