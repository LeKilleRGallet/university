\documentclass[12pt,letterpaper]{article}
\usepackage{amsmath}

\title{Existencia de la economia dual en Colombia}
\author{Augusto Rico}
\date{\today}

\begin{document}
\maketitle
\section{Introduccion}

\begin{flushleft}   


el siguiente articulo esta enfocado en comprobar empiricamente la existencia de una economia dual en Colombia tal como plantea Lewis(1954).
Esta hipotesis se basa en la consideracion por parte del FMI(2020) de Colombia como una economia en via de Desarrollo, por ende si esto es cierto 
y basandonos en la tesis de W.Arthur Lewis Colombia debe tener tambien las caracteristicas de una economia Dual.\\
~\\
las economias en via de desarrollo tal como plantea Lewis(1954) estan caracterizadas por la coexistencia de dos tipos de economia, la moderna caracterizada
por altos niveles relativos de productividad ya que son intensivas en bienes de capital, y la economia tradicional caracterizada por bajos niveles de productividad
debido no existencia de bienes de capital en este sector, lo que implica bajos salarios para los trabajadores.\\
~\\
En Colombia este hecho se puede ver reflejado principalmente en las diferencias economicas entre las ciudades principales y las zonas rurales, que como se expondra
las ciudades princiaples estan caracterizadas por tener sectores modernos e industriales con altos niveles relativos de ingresos y bienestar, y las zonas rurales
apartadas del centro caracterizadas por la escazes de bienes de capital y un fuerte sector rural con baja productividad sostenido por uso intensivo de mano de obra con salarios de subsitencia.\\
~\\



\end{flushleft}



\end{document}