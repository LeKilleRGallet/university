\documentclass{article}
\usepackage{amsmath}

\title{Derecho de peticion}
\author{ }
\date{\today}



\begin{document}
\maketitle

\section{Asunto}
\begin{flushleft}


    Derecho de Petición. Artículo 23 de la Constitución Política y Ley 1755 de 2015

    Respetados señores del congreso,

    Yo, Augusto David Rico Dautt, identificado con cedula 1140900470, de conformidad con lo establecido en el artículo 23 de la Constitución Política, en concordancia con la Ley 1755 de 2015, comedidamente me permito presentar la petición que más adelante se describe.

    
\end{flushleft}



\section{Peticion}
\begin{flushleft}

    pido por favor se me sea suministrada una base de datos con los siguientes datos:\\
    1. Nombre de todos los congresistas elegidos apartir de la constitucion del 91\\
    2. tipo de eleccion en la que se presento el congresista. ejemplo: senado, camara por bogota, camara por tolima, camara afro, senado indigena, curul especial de paz\\
    3. legislaturas en las que estuvo presente el congresista\\
    4. partido politico en cada una de sus legislaturas\\
    5. nombre de las comisiones en las que estuvo presente el congresista especificando la legislatura\\
    6. si el congresista no logro resultar releecto en alguna eleccion y cuales elecciones o si por el contrario decidio no volver a presentarse\\
    7. numeros de leyes estatutarias impulsadas por el congresista en cada legislatura\\
    8. numero de leyes organicas impulsadas por el congresista en cada legislatura\\
    9. numero de leyes ordinarias exeptuando leyes de honores impulsadas por el congresista en cada legislatura\\
    10. numero de leyes de honores impulsadas por el congresista en cada legislatura\\
    11. numero de mociones de censura impulsadas por el congresista en cada legislatura\\
    12. numero de debates citados por el congresista en cada legislatura\\
    13. numero de proyectos de ley presentados por el congresista en cada legislatura\\


\end{flushleft}

\section{Finalidad}
\begin{flushleft}
    realizar un estudio econometrico para comprender que es lo que hace a los congresistas ser reelegidos.
\end{flushleft}

\section{Notificacion}

\begin{flushleft}
    Por favor notifique al siguiente medio\\
    Correo: arico@unal.edu.co
\end{flushleft}

\begin{flushleft}
    cordialmente, Augusto Rico
    cc : 1140900470
    telefono: 3187352165
\end{flushleft}
\end{document}