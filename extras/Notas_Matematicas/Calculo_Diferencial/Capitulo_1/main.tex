\documentclass[11pt]{article}
\usepackage{UF_FRED_paper_style}
\usepackage{SMath}
% \usepackage{APA}
\onehalfspacing
\setlength{\droptitle}{-5em} %% Don't touch

\title{Una introduccion \textit{no}-informal a las matematicas.
%%\thanks{thanks}
}

% AUTHORS:
\author{Augusto Rico\\% Name author
    \href{mailto:arico@unal.edu.co}{\texttt{arico@unal.edu.co}} %% Email author 1 
% \and Second Author\\% Name author
%     \href{mailto:secondauthor@email.com}{\texttt{secondauthor@email.com}} %% Email author 2
    }

\date{\today}

\begin{document}
\maketitle
% %%%%%%%%%%%%%%%%%%%%%%%%%%%%%%%%%%%%%%%%%%%%%%%%%%%%%%%%%%
% %%%%%%%%%%%%%%%%%%%%%%%%%%%%%%%%%%%%%%%%%%%%%%%%%%%%%%%%%%
% BODY OF THE DOCUMENT
% %%%%%%%%%%%%%%%%%%%%%%%%%%%%%%%%%%%%%%%%%%%%%%%%%%%%%%%%%%
% %%%%%%%%%%%%%%%%%%%%%%%%%%%%%%%%%%%%%%%%%%%%%%%%%%%%%%%%%%

% --------------------
\section{Propiedades basicas de los numeros}
% --------------------
\begin{flushleft}
    para iniciar un estudio de las matematicas es necesario de antemano asumir ciertos conceptos fundamentales
    que seran demostrados a \textit{posteriori} en el estudio de las matematicas como es el concepto de la operacion de la suma
    \begin{definition}
        existe una operacion binaria llamada \textbf{«suma»} representada con el simbolo $'+'$ mediante la cual se pueden operar dos numeros
        cualesquiera obteniendo como resultado otro numero.
    \end{definition}
    \begin{example}
        sean $a,b,c$ tres numeros cualesquiera tendremos que $S(a,b) \to c$ que se lee como 
        \textit{la suma de $a$ y $b$ es $c$} y se puede representar preferentemente como $a+b=c$
        y se puede leer como \textit{ $a$ «mas» $b$ «es igual a»  $c$}
    \end{example}
    habiendo definido la existencia de la suma podemos mostrar algunas de sus propiedades que despues demostraremos que son ciertas,
    no obstante por el momento solo las asumiremos como verdaderas, estas propiedades son:
    \begin{dogma}[Ley conmutativa de la suma]
        para cualquier numero $a$ y $b$ se tiene que $a+b=b+a$
    \end{dogma}
    \begin{dogma}[Existencia de elemento neutro en la suma]
        Existe un numero $I_s$ llamado \textbf{«elemento neutro de la suma»} tal que para cualquier numero $a$ se tiene que $a+I_s=a$
    \end{dogma}
    \begin{dogma}[Existencia del inverso en la suma]
        para cualquier numero $a$ existe un numero $-a$ llamado \textbf{«inverso de a»} tal que $a+(-a)=I_s$
    \end{dogma}
    \begin{dogma}[Ley asociativa de la suma]
        para cualquier numero $a$, $b$ y $c$ se tiene que $(a+b)+c=a+(b+c)$
    \end{dogma}
    habiendo definido la suma y sus propiedades podemos definir otra operacion binaria llamada \textbf{«multiplicacion»} 
    \begin{definition}
        existe una operacion binaria derivada de la suma llamada \textbf{«multiplicacion»} representada con el simbolo $'\times'$
        mediante la cual se pueden operar dos numeros cualesquiera obteniendo como resultado otro numero.\\
    \end{definition}
    \begin{example}
        sean $a,b,c$ tres numeros cualesquiera tendremos que $M(a,b) \to c$ que se lee como 
        \textit{la multiplicacion de $a$ y $b$ es $c$} y se puede representar preferentemente como $a\times b=c$
        y se puede leer como \textit{ $a$ «por» $b$ «es igual a»  $c$}.\\
        donde $a\times b$ es equivalente a $a+a+a+\cdots+a$ donde $a$ se repite $b$ veces.
    \end{example}

    
\end{flushleft}


\newpage

% %%%%%%%%%%%%%%%%%%%%%%%%%%%%%%%%%%%%%%%%%%%%%%%%%%%%%%%%%%
% %%%%%%%%%%%%%%%%%%%%%%%%%%%%%%%%%%%%%%%%%%%%%%%%%%%%%%%%%%
% REFERENCES SECTION
% %%%%%%%%%%%%%%%%%%%%%%%%%%%%%%%%%%%%%%%%%%%%%%%%%%%%%%%%%%
% %%%%%%%%%%%%%%%%%%%%%%%%%%%%%%%%%%%%%%%%%%%%%%%%%%%%%%%%%%
\medskip

% \bibliography{references.bib} 

\newpage

\end{document}