\documentclass[11pt]{article}
\usepackage{UF_FRED_paper_style}
\usepackage{lipsum}
\onehalfspacing
\allowdisplaybreaks[1]

\setlength{\droptitle}{-5em} %% Don't touch

\title{Parcial 1 Microeconomia 3\\
    \large Profesor: Salomon Bechara Senior
}

\author{Augusto Rico\\CC. 1140900470\\
    \href{mailto:arico@unal.edu.co}{\texttt{arico@unal.edu.co}}
    }

\date{\today}

\begin{document}

\setstretch{.8} %% Don't touch
\maketitle


% %%%%%%%%%%%%%%%%%%%%%%%%%%%%%%%%%%%%%%%%%%%%%%%%%%%%%%%%%%
% %%%%%%%%%%%%%%%%%%%%%%%%%%%%%%%%%%%%%%%%%%%%%%%%%%%%%%%%%%
% BODY OF THE DOCUMENT
% %%%%%%%%%%%%%%%%%%%%%%%%%%%%%%%%%%%%%%%%%%%%%%%%%%%%%%%%%%
% %%%%%%%%%%%%%%%%%%%%%%%%%%%%%%%%%%%%%%%%%%%%%%%%%%%%%%%%%%

\section{Punto 1}
% --------------------
\begin{flushleft} 
    dado que la solucion del monopolista es $MR=MC$ pero como en equilibrio $\epsilon\in(-1,0)$ sabemos entonces que 
    $MR<0$ no obstane como $MC>0$ entonces $MR$ no podra ser igual a $MC$ por lo que el
    monopolista debe operar unicamente en la parte donde $MR>0$ y para ello entonces se debe trabajar en la parte elastica de la demanda
    para que $\epsilon>1$ y de esa forma logre tener solucion de maximizacion de beneficios. 
\end{flushleft}

\section{Punto 2}
% --------------------
\begin{flushleft}    
    \begin{align*}
        \pi = (100-Q)Q-\frac{4Q^2}{2} \to 100Q-Q^2-\frac{4Q^2}{2}\\
        \frac{\partial \pi}{\partial Q} = 0 \to 100-2Q-4Q=0\\ 100-6Q=0 \to Q=\frac{100}{6}=\frac{50}{3}\\
    \end{align*}
    habiendo obtenido las cantidades las remplazamos en el precio demandado para obtener el precio de monopolio:
    \begin{align*}
        P=100-\frac{50}{3}=\frac{250}{3}\\
    \end{align*}
    calculamos los excedentes
    \begin{align*}
        CS=\frac{\left(100-\frac{250}{3}\right)\frac{50}{3}}{2}=\frac{277\frac{7}{9}}{2}=138.89\\
        PS=\int_{0}^{50/3}100-x~dx - \int_{0}^{50/3} 4x ~dx = \frac{8750}{9} = 972.22
    \end{align*}
    Calculamos los beneficios del monopolista para las cantidades dadas.
    \begin{align*}
        \pi = \frac{250}{3}*\frac{50}{3}-\frac{4\left(\frac{50}{3}\right)^2}{2}=\frac{2500}{3}=833.33
    \end{align*}
\end{flushleft}
\section{Punto 3}
% --------------------
\begin{flushleft}
    si el gobierno regula al monopolio a producir el nivel socialmente eificente implica obligarlos a que $P=MC$, por lo que tendriamos que:
    \begin{align*}
        100-Q=10 \to Q=90
    \end{align*}
    por lo que el monopolista produciria 90 unidades, y el precio seria $100-90=10$, por lo que el monopolista
    obtendria un beneficio de $\pi=10*90-(50+10(90))=-50$ por lo que el monopolista deberia incurrir en perdidas, y por
    ende va a tener incentivos para cerrar la compañia, dado que no producira beneficios.
    si el gobierno al ver esto quiere regular de tal forma que el monopolio produzca la mayor
    cantidad de bienes sin incurrir en perdidas entonces debe resolver la siguiente ecuacion:
    \begin{align*}
        (100-Q)Q-(50+10Q)=0\\
        100Q-Q^2-50-10Q=0\\
        -Q^2+90Q-50=0\\
        Q=\frac{-90\pm\sqrt{90^2-4(-1)(-50)}}{-2}\\
        Q=\frac{-90\pm\sqrt{8100-200}}{-2}\\
        Q=\frac{-90\pm\sqrt{7900}}{-2}\\
        Q=\frac{-90\pm10\sqrt{79}}{-2}\\
        Q=\frac{-10(9\pm\sqrt{79})}{-2}\\
        Q=5(9\pm\sqrt{79})\\
        Q_1=5(9+\sqrt{79})\\
        Q_2=5(9-\sqrt{79})\\~\\
        \text{como } Q_1>Q_2 \text{ entonces } Q=Q_1=5(9+\sqrt{79})\\~\\
        \text{y el precio sera:}\\~\\
        P=100-Q \to 100-5(9+\sqrt{79}) \\
        P=100-45-5\sqrt{79}\\
        P=55-5\sqrt{79}\\
    \end{align*}
\end{flushleft}


\section{Punto 4}
% --------------------
\begin{flushleft}
    \begin{itemize}
        \item Cantidades producidas: En todas las formas de discriminación de precios, la firma puede producir una cantidad distinta a la cantidad que produciría en un mercado de competencia perfecta.
        Esto se debe a que la firma tiene poder de mercado y por ende puede fijar las cantidades de bienes totales en la economia, normalmente buscando maximizar su beneficio.

        \item Excedentes del consumidor y del productor: La discriminación de precios en cualquiera de sus formas puede aumentar o disminuir el excedente del consumidor y del productor, en comparación con un mercado de competencia perfecta. En general, la discriminación de precios puede aumentar el excedente del productor y disminuir el excedente del consumidor, lo que significa que la firma captura una porción mayor del excedente total, desmejorando el bienestar social.
        
        \item Beneficios de la firma: La discriminación de precios puede aumentar los beneficios de la firma en comparación con un mercado de competencia perfecta.
        
    \end{itemize}
\end{flushleft}

\section{Punto 5}
% --------------------
\begin{flushleft}
    sabiendo que $Q_A+Q_B=Q$ podemos tener la siguiente funcion de beneficios para nuestro monopolista $\pi=(P_AQ_A+P_BQ_B)-C(Q_A+Q_B)$,
    o lo que es igual a 
    \begin{align*}
        \pi=100Q_A-Q_A^2+100Q_B-2Q_B^2-\frac{(Q_A+Q_B)^2}{2}
    \end{align*}
    ecuacion en la cual podemos obtener las condiciones de primer orden para obtener las cantidades obtimas de cada tipo de consumidor.
    \begin{align*}
        \frac{\partial \pi}{\partial Q_A}=0 \to 100-2Q_A-Q_A-Q_B=0\to100-3Q_A=Q_B\\
        \frac{\partial \pi}{\partial Q_B}=0 \to 100-4Q_B-Q_B-Q_A=0\to100-5Q_B=Q_A\\~\\
    \end{align*}
    remplazamos $Q_A$ en la primera ecuacion para resolver el sistema, obteniendo:
    \begin{align*}
        100-3(100-5Q_B)=Q_B \to 100-300+15Q_B=Q_B\\~\\
        200=14Q_B \to Q_B=\frac{200}{14}=\frac{100}{7}\\
    \end{align*}
    remplazamos lo obtenido en $Q_B$ para obtener $Q_A$:
    \begin{align*}
        Q_A=100-5\left(\frac{100}{7}\right)=28.57\\
    \end{align*}
    remplazamos para obtener cada precio de cada tipo de consumidor:
    \begin{align*}
        P_A=100-(28.57)=71.43\\
        P_B=100-2\left(\frac{100}{7}\right)=71.43\\
    \end{align*}
    remplazamos para obtener la ganancia del monopolista:
    \begin{align*}
        \pi=71.43*28.57+71.43*\frac{100}{7}-\frac{(28.57+\frac{100}{7})^2}{2}=2142.88\\
    \end{align*}
    calculamos las elasticidades de cada tipo de consumidor:
    \begin{align*}
        \epsilon_A=\frac{P_A}{Q_A}\frac{\partial Q_A}{\partial P_A}=\frac{71.43}{28.57}*-1 =-2.5\\
        \epsilon_B=\frac{P_B}{Q_B}\frac{\partial Q_B}{\partial P_B}=\frac{71.43}{\frac{100}{7}}*-\frac{1}{2}=-2.5\\
    \end{align*}
    por lo que notamos que ambos tipos de consumidor tienen la misma elasticidad.\\~\\
    si el monopolista no pudiera discriminar entonces: 
    \begin{align*}
        Q = 100-p+50-\frac{p}{2}=-\frac{3p}{2}+150
    \end{align*}
    y calculando la demanda inversa obtenemos:
    \begin{align*}
        p=100-\frac{2q}{3}\\
    \end{align*}
    remplazamos en la funcion de beneficios para obtener la funcion de beneficios del monopolista sin discriminacion:
    \begin{align*}
        \pi=(-\frac{2q}{3}+100)q-\frac{q^2}{2}\\
        \frac{\partial \pi}{\partial q} = 0 \to -\frac{4q}{3}+100-q=0\\
        -\frac{7q}{3}=-100 \to q=\frac{300}{7}\\~\\
        \text{remplazando en el precio:}\\
        p=-\frac{2(\frac{300}{7})}{3}+100=\frac{500}{7}=71.43\\~\\
        \text{calculando el beneficio:}\\
        \pi = 71.43*\left(\frac{300}{7}\right)-\frac{\left(\frac{300}{7}\right)^2}{2}=2142.86\\
    \end{align*}
    notando que el beneficio es identico al obtenido con discriminacion de precios.\\~\\
\end{flushleft}


\section{Puntos antiguos hechos antes de la correccion}
\subsection{Punto 2 - Antiguo}
% --------------------
\begin{align*}
    \pi = \frac{100}{q}q-\frac{4q^2}{2}\to 100-\frac{4q^2}{2}\\
    \frac{\partial \pi}{\partial q} = 0 \to -4q=0 \to q=0\\
\end{align*}
\begin{flushleft}
    dado que $p(0)$ no existe es necesario evaluar el \textit{liimite} de la funcion en $0^+$
\end{flushleft}
\begin{align*}
    \lim _{x\to 0^+}\left(p(x)\right) \to \lim _{x\to 0^+}\left(\frac{100}{x}\right) \to \frac{100}{0^+}= \infty\\
\end{align*}
\begin{flushleft}
    por lo que como el monopolista no producira el precio de mercado tendera al infinito.\\
    dado que $q=0$ los exedentes tanto del productor como del consumidor seran 0, ya que $ \forall x: \int_{0}^{0}x~dx=0$, no obstante el monopolista obtendra una ganancia de $100$,
    dado que $100-\frac{4(0)^2}{2}=100$.
\end{flushleft}
\subsection{Punto 3 -Antiguo}
% --------------------
\begin{flushleft}
    si el gobierno regula al monopolio a producir el nivel socialmente eificente implica obligarlos a que $P=MC$, por lo que tendriamos que:
    \begin{align*}
        \frac{100}{q}=10 \to q=\frac{100}{10}=10\\
    \end{align*}
    por lo que el monopolista tendria que producir 10 unidades, y el precio de mercado seria $100/10=10$,
    por lo que el monopolista obtendria una ganancia de $100-(50+10(10))=-50$ por lo que el monopolista deberia incurrir en perdidas, y 
    por ende va a tener incentivos para cerrar la compañia, dado que no producira beneficios.\\
    si el gobierno al ver esto quiere regular de tal forma que el monopolio produzca la mayor cantidad de bienes
    sin incurrir en perdidas entonces debe resolver la siguiente ecuacion: $100-(50+10(q))=0$ obteniendo entonces que $q=5$ y $p=20$.
\end{flushleft}

\end{document}