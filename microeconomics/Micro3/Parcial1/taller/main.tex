\documentclass[11pt]{article}
\usepackage{UF_FRED_paper_style}
\usepackage{lipsum}
\onehalfspacing

\setlength{\droptitle}{-5em} %% Don't touch

\title{Taller-parcial Complementario
}

\author{Augusto Rico\\
    \href{mailto:arico@unal.edu.co}{\texttt{arico@unal.edu.co}}}

\date{\today}

\begin{document}

\setstretch{.8} %% Don't touch
\maketitle


% %%%%%%%%%%%%%%%%%%%%%%%%%%%%%%%%%%%%%%%%%%%%%%%%%%%%%%%%%%
% %%%%%%%%%%%%%%%%%%%%%%%%%%%%%%%%%%%%%%%%%%%%%%%%%%%%%%%%%%
% BODY OF THE DOCUMENT
% %%%%%%%%%%%%%%%%%%%%%%%%%%%%%%%%%%%%%%%%%%%%%%%%%%%%%%%%%%
% %%%%%%%%%%%%%%%%%%%%%%%%%%%%%%%%%%%%%%%%%%%%%%%%%%%%%%%%%%

% --------------------
\section{Punto 1}
% --------------------
\subsection{monopolio discriminador de grado 3}
\begin{flushleft}
    sabemos que la condicion de maximizacion para un monopolio es:
    \begin{align*}
        MR_i = MC \to 
        A_i-2B_iQ_i = \frac{A_1+A_2}{2}\\
        2A_i-4B_iQ_i = A_1+A_2\\
        Q_i=\frac{A_1+A_2-2A_i}{-4B_i}\\
    \end{align*}
    para el consumidor tipo $1$:
    \begin{align*}
        Q_1^m=\frac{A_1+A_2-2A_1}{-4B_1} \to -\frac{A_2-A_1}{4B_1}\\
        P^m_1=A_1-B_1\left(-\frac{A_2-A_1}{4B_1}\right)\\
        P^m_1=A_1+\frac{A_2-A_1}{4} \to \frac{3A_1+A_2}{4}\\
    \end{align*}
    para el consumidor tipo $2$:
    \begin{align*}
        Q_2^m=\frac{A_1+A_2-2A_2}{-4B_2} \to -\frac{A_1-A_2}{4B_2}\\
        P^m_2=A_2-B_2\left(-\frac{A_1-A_2}{4B_2}\right)\\
        P^m_2=A_2+\frac{A_1-A_2}{4} \to \frac{3A_2+A_1}{4}\\
    \end{align*}
\end{flushleft}
\subsection{monopolio sin capacidad de discriminar}
\begin{flushleft}
    \begin{align*}
        P=P_1+P_2 \to A_1-B_1Q_1+A_2-B_2Q_2 \to \underbrace{\sum A_i}_{A}-\underbrace{\sum B_iQ_i}_{BQ} \to P=A-BQ\\
    \end{align*}
    con la ecuacion de demanda, aplicamos la derivada a la utilidad del monopolista y obtenemos las cantidades optimas:
    \begin{align*}
        MR = MC \to A-2BQ = \frac{A}{2}\\
        2A-4BQ = A\\
        Q^m=\frac{A}{4B}\\
    \end{align*}
    cantidades que generan un precio:
    \begin{align*}
        P^m=A-B\left(\frac{A}{4B}\right)\\
        P^m=A-\frac{A}{4} \to \frac{3A}{4}\\
    \end{align*}
\end{flushleft}
\section{Punto 2}
\subsection{monopolio discriminador de grado 3}
\begin{flushleft}
    \begin{align*}
        MR_i = MC \to A_i-2B_iQ_i = \frac{A_1-A_2}{2}\\
        2A_i-4B_iQ_i = A_1-A_2\\
        Q_i=\frac{A_1-A_2-2A_i}{-4B_i}\\
    \end{align*}
    para el consumidor tipo $1$:
    \begin{align*}
        Q_1^m=\frac{A_1-A_2-2A_1}{-4B_1} \to -\frac{-A_1-A_2}{4B_1}\\
        P^m_1=A_1-B_1\left(-\frac{-A_1-A_2}{4B_1}\right)\\
        P^m_1=A_1+\frac{-A_1-A_2}{4} \to \frac{3A_1-A_2}{4}\\
    \end{align*}
    para el consumidor tipo $2$:
    \begin{align*}
        Q_2^m=\frac{A_1-A_2-2A_2}{-4B_2} \to -\frac{A_1-3A_2}{4B_2}\\
        P^m_2=A_2-B_2\left(-\frac{A_1-3A_2}{4B_2}\right)\\
        P^m_2=A_2+\frac{A_1-3A_2}{4} \to \frac{5A_1-3A_2}{4}\\
    \end{align*}
\end{flushleft}
\subsection{monopolio sin capacidad de discriminar}
\begin{flushleft}
    \begin{align*}
        P=P_1+P_2 \to A_1-B_1Q_1+A_2-B_2Q_2 \to \underbrace{\sum A_i}_{A}-\underbrace{\sum B_iQ_i}_{BQ} \to P=A-BQ\\
    \end{align*}
    con la ecuacion de demanda, aplicamos la derivada a la utilidad del monopolista y obtenemos las cantidades optimas:
    \begin{align*}
        &MR = MC \to A-2BQ = \frac{A_1-A_2}{2}\\
        &2A-4BQ = A_1-A_2\\
        &Q^m=\frac{A_1-A_2-2A}{-4B} \to \frac{A_1-A_2-2A_1-2A_2}{-4B} \to \frac{A_1+3A_2}{4B}\\
    \end{align*}
    cantidades que generan un precio:
    \begin{align*}
        &P^m=A-B\left(\frac{A_1+3A_2}{4B}\right)\\
        &P^m=A-\frac{A_1+3A_2}{4}\\
        &P^m = \frac{4A-A_1-3A_2}{4} \to \frac{4A_1+4A_2-A_1-3A_2}{4}\\
        &P^m = \frac{3A_1+A_2}{4}\\
    \end{align*}
\end{flushleft}
\subsection{comparacion}
\begin{flushleft}
    dado que en el caso 2 el monopolista tiene menores costos marginales de produccion, sus soluciones se dan por mayor cantidad de cantidades producidad en todos
    los casos, y por ende, menores precios para los consumidores.
\end{flushleft}
\section{Punto 3}
\begin{flushleft}

\end{flushleft}
\section{Punto 4}
\begin{flushleft}
    el problema que busca resolver un monopolista legal siempre sera el de obtener las cantidades optimas donde logre maximizar su beneficio,
    para ello, el monopolista entonces debera resolver:
    \begin{align*}
        \max_q \pi(q) = p(q)q-c(q)\\
    \end{align*}
    no obstante, para que este pueda obtener una solucion derivando se debe satisfacer que, $\pi''(q)<0$,
    que esto se puede satisfacer si $c(\cdot)$ es convexa y el ingreso marginal tiene pendiente negativa, bajo estas condiciones, y como tenemos demandas lineales
    podemos derivar e igualar a cero y se deberia obtener un maximo.
    \begin{align*}
        \pi'(q) = p'(q)q+p(q)-c'(q) = 0\\
    \end{align*}
    despejando obtenemos que:
    \begin{align*}
        \underbrace{p'(q)q+p(q)}_{MR} = \underbrace{c'(q)}_{MC}\\
    \end{align*}
    factorizando $p(q)$ en la parte derecha podemos obtener que:
    \begin{align*}
        p(q)\left[1+\frac{qp'(q)}{p(q)}\right]=c'(q)\\
    \end{align*}
    $p'(q)$ puede representarse tambien como $\frac{dp(q)}{dq}$, por lo que podemos reescribir la ecuacion como:
    \begin{align*}
        p(q)\left[1+\frac{q}{p(q)}\frac{dp(q)}{dq}\right]=c'(q)\\
    \end{align*}
    sabemos que $\epsilon = \frac{p(q)}{q}\frac{dq}{dp(q)}$ por lo que podemos reescribir nuestra ecuacion como:
    \begin{align*}
        p(q)\left[1+\frac{1}{\epsilon}\right]=c'(q)\\
    \end{align*}
    no obstante, en equilibrio $-1 < \epsilon < 0$ por lo que $MR<0$, y dado que $c(\cdot)$ es convexa, $MC>0$, por lo que no se podria satisfacer la solucion,
    por esto el monopolista debe trabajar en la parte elastica de la curva de demanda.

\end{flushleft}
\end{document}