\documentclass[11pt]{article}
\usepackage{UF_FRED_paper_style}
\usepackage{lipsum}
\onehalfspacing

\setlength{\droptitle}{-5em} %% Don't touch

\title{preguntas Ricketts
}

\author{Augusto Rico\\
    \href{mailto:arico@unal.edu.co}{\texttt{arico@unal.edu.co}}}

\date{\today}

\begin{document}

\setstretch{.8} %% Don't touch
\maketitle


% %%%%%%%%%%%%%%%%%%%%%%%%%%%%%%%%%%%%%%%%%%%%%%%%%%%%%%%%%%
% %%%%%%%%%%%%%%%%%%%%%%%%%%%%%%%%%%%%%%%%%%%%%%%%%%%%%%%%%%
% BODY OF THE DOCUMENT
% %%%%%%%%%%%%%%%%%%%%%%%%%%%%%%%%%%%%%%%%%%%%%%%%%%%%%%%%%%
% %%%%%%%%%%%%%%%%%%%%%%%%%%%%%%%%%%%%%%%%%%%%%%%%%%%%%%%%%%

% --------------------
\section{ ¿Qué es la teoría del principal y el agente, y cómo aborda el problema del cumplimiento contractual?}
% --------------------
\begin{flushleft}
    La teoría del principal y el agente sostiene que cuando una persona contrata a alguien para realizar una tarea, pero tienen diferentes metas y objetivos, la persona contratada puede actuar en su propio interés en lugar del interés de quien los contrató, lo que puede generar problemas en el contrato.

    Para evitar esta situación, la teoría sugiere la implementación de incentivos, como el pago adicional si se cumplen ciertas metas, o la supervisión del trabajo realizado. Estos incentivos ayudan a que la persona contratada trabaje para el mismo objetivo que la persona que los contrató.

    Es igualmente importante que el contrato sea claro y completo, de manera que ambas partes sepan qué se espera de ellos y las consecuencias en caso de incumplimiento. Esto reduce la posibilidad de problemas en el contrato.
\end{flushleft}

\section{¿Cómo impacta la presencia o ausencia de confianza y buena reputación en el diseño óptimo de un contrato?}
\begin{flushleft}
    La presencia o ausencia de confianza y buena reputación tienen un gran impacto en el diseño óptimo de un contrato. Cuando dos personas o empresas confían la una en la otra y tienen buenas reputaciones, pueden hacer contratos más simples y fáciles porque confían en que ambas partes cumplirán con lo prometido. Por ejemplo, si una empresa contrata a un proveedor con el que ya ha trabajado y sabe que hace un buen trabajo, puede hacer un contrato sencillo con pocas condiciones, lo que reduce el costo de transacción y mejora la eficiencia de la negociación.

    Por otro lado, si una o ambas partes no tienen una buena reputación o no se confían mutuamente, es probable que se necesiten más términos específicos en el contrato, lo que aumenta el costo de transacción. Además, es posible que se necesiten medidas de monitoreo y cumplimiento más rigurosas, lo que también aumenta el costo de transacción. Por ejemplo, si una empresa contrata a un proveedor desconocido o que tiene mala reputación, necesita un contrato con muchas condiciones específicas para asegurarse de que el proveedor haga lo que se espera.

    La reputación de las personas o empresas afecta su comportamiento y puede ser especialmente importante en contratos que involucran a agentes. Si alguien tiene una buena reputación, es más probable que haga lo correcto porque sabe que su reputación es importante para su negocio. Los agentes con buena reputación son más propensos a actuar de manera ética y cumplir con las obligaciones acordadas. Por el contrario, si alguien no tiene una buena reputación, es posible que no le importe tanto hacer lo correcto y pueda comportarse de manera deshonesta. Los agentes sin reputación son menos confiables y pueden ser más propensos a actuar de manera oportunista. Es probable que el agente con reputación tenga más incentivos para actuar de manera ética y cumplir con las obligaciones acordadas debido a que desea proteger su buena reputación, mientras que el agente sin reputación puede ser más propenso a actuar de manera oportunista si cree que puede salirse con la suya sin consecuencias negativas.

    El dilema del prisionero es un juego que se utiliza para ilustrar este problema. En un juego no repetido, ambas partes pueden tener incentivos para actuar de manera oportunista y no cumplir con las obligaciones acordadas si no confían el uno en el otro o no tienen una buena reputación. Por otro lado en un juego repetido, la posibilidad de un futuro juego puede incentivar a ambas partes a actuar de manera ética y construir una buena reputación a lo largo del tiempo.
\end{flushleft}

\section{¿En qué se diferencia la tradición de costos de transacción de Coase/Williamson del enfoque de razonamiento económico neoclásico para el diseño de contratos?}

\begin{flushleft}
    La tradición de costos de transacción de Coase/Williamson se diferencia del enfoque de razonamiento económico neoclásico para el diseño de contratos en varios aspectos clave.

\begin{itemize}
    \item la tradición de costos de transacción se centra en el análisis de los costos que surgen en las transacciones económicas, tales como los costos de búsqueda, negociación, monitoreo y aplicación de los contratos. Esta tradición reconoce que los costos de transacción pueden ser significativos y pueden afectar el diseño y la eficiencia de los contratos.

    Por otro lado, el enfoque neoclásico para el diseño de contratos se basa en la suposición de que los contratos son completos y que los costos de transacción son cero. En este enfoque, se espera que los contratos sean capaces de especificar todas las contingencias posibles y de garantizar que ambas partes obtengan el máximo beneficio posible.
    
     \item la tradición de costos de transacción también reconoce que los contratos pueden ser incompletos y que, por lo tanto, pueden requerir la confianza y la cooperación entre las partes para ser efectivos. Esta perspectiva esencialmente reconoce la importancia de las relaciones y la confianza en las transacciones económicas, lo que puede ser especialmente importante en contextos de incertidumbre.
    
    Por otro lado, el enfoque neoclásico supone que las relaciones entre las partes son impersonales y que la confianza no es necesaria para la eficiencia de los contratos.
    
\end{itemize}
En resumen, la tradición de costos de transacción de Coase/Williamson se diferencia del enfoque neoclásico para el diseño de contratos en que reconoce la importancia de los costos de transacción y la incompletitud de los contratos, así como la importancia de las relaciones y la confianza en las transacciones económicas.
\end{flushleft}

\section{¿Cuáles son los principales problemas abordados por la tradición de los costos de transacción y cómo la racionalidad limitada (acotada) impacta la solución de los problemas contractuales? }

\begin{flushleft}
    Un problema importante que aborda la tradición de los costos de transacción es la existencia de asimetrías de información entre las empresas. Esto se refiere a la situación en la que una empresa tiene más información que otra empresa en la negociación y ejecución de un contrato. Las asimetrías de información pueden llevar a un resultado subóptimo en la asignación de recursos y en la creación de valor.

    La racionalidad limitada o acotada también tiene un impacto significativo en la solución de los problemas contractuales. La racionalidad limitada se refiere a la limitación de la capacidad humana para procesar y analizar información de manera completa y coherente. Como resultado, los individuos y las empresas pueden tomar decisiones que no son completamente racionales o que no maximizan sus beneficios.

    En el contexto de los problemas contractuales, la racionalidad limitada puede llevar a la selección de contratos subóptimos o a la incapacidad de llegar a un acuerdo que maximice el beneficio mutuo. Además, la racionalidad limitada puede hacer que las empresas se centren en soluciones a corto plazo en lugar de soluciones a largo plazo, lo que puede limitar el potencial de creación de valor. Por lo tanto, es importante que las empresas tomen en cuenta la racionalidad limitada en su toma de decisiones y en la negociación y ejecución de contratos.
\end{flushleft}

\section{¿De qué manera puede el problema principal-agente y los contratos incompletos explicar las características internas de las empresas, como las estructuras jerárquicas y los incentivos para el comportamiento cooperativo?}

\begin{flushleft}
    La combinación del problema principal-agente y los contratos incompletos puede explicar las características internas de las empresas de varias maneras. Por ejemplo, las estructuras jerárquicas pueden ser una forma en que los propietarios de empresas intentan minimizar los problemas de agencia al establecer una cadena de mando y control. Al hacerlo, los propietarios pueden tener más control sobre el comportamiento de los gerentes y otros empleados, lo que puede ayudar a alinear los objetivos de la empresa con los objetivos de los propietarios.

    Además, los incentivos para el comportamiento cooperativo pueden ser una forma en que los propietarios de empresas intentan resolver los problemas de agencia. Por ejemplo, los propietarios pueden proporcionar incentivos financieros, como bonificaciones, a los gerentes y empleados que logran ciertos objetivos importantes para la empresa. Esto puede motivar a los gerentes y empleados a trabajar en colaboración para lograr estos objetivos, en lugar de perseguir sus propios intereses individuales.

    Por lo tanto, el problema principal-agente y los contratos incompletos son conceptos importantes para comprender las características internas de las empresas, como las estructuras jerárquicas y los incentivos para el comportamiento cooperativo. Al comprender estos conceptos, los propietarios de empresas pueden tomar medidas para minimizar los problemas de agencia y promover el éxito empresarial.
\end{flushleft}

\end{document}