\documentclass[11pt]{article}
\usepackage{UF_FRED_paper_style}
\usepackage{lipsum}
\onehalfspacing

\setlength{\droptitle}{-5em} %% Don't touch

\title{A Hypothesis of Wealth-Maximizing Norms: Evidence from the Whaling Industry
}

\author{Augusto Rico\\
    \href{mailto:arico@unal.edu.co}{\texttt{arico@unal.edu.co}}}

\date{\today}

\begin{document}

\setstretch{.8} %% Don't touch
\maketitle
% --------------------
\section{¿Cuál es la hipótesis principal presentada en el artículo sobre las normas que maximizan la riqueza en la industria ballenera? ¿Cómo se relaciona esta hipótesis con la teoría económica tradicional?}
% --------------------
\begin{flushleft}
    El artículo resalta la importancia de las normas sociales en el comportamiento económico, utilizando el caso de la industria ballenera del siglo XIX como ejemplo. A pesar de los incentivos para cooperar, los cazadores de ballenas a menudo no lo hacían debido a la falta de confianza y comunicación, lo que llevaba a perder oportunidades de maximizar su riqueza.

El autor argumenta que se establecieron normas sociales que fomentaban la cooperación y el intercambio de información, como compartir avistamientos de ballenas y repartir las ganancias de manera equitativa. La evidencia respalda la idea de que estas normas facilitaron la cooperación, lo cual resultó en una mayor riqueza tanto a nivel individual como grupal.

Esto desafía la suposición de la teoría económica tradicional de que los individuos actúan de forma egoísta para maximizar su utilidad. Al reconocer la influencia de las normas sociales, se amplía nuestra comprensión del comportamiento económico y se destaca el impacto de las interacciones sociales en nuestras decisiones.

Además, estas ideas están relacionadas con la teoría de juegos, donde las normas sociales pueden alterar los incentivos y las estrategias, generando resultados más favorables tanto para los individuos como para el grupo.
\end{flushleft}

\section{Según el artículo, ¿cómo se desarrollaron y evolucionaron las normas en la industria ballenera para maximizar la riqueza? ¿Qué factores socioculturales y económicos influyeron en este proceso?}

\begin{flushleft}
    Las normas en la industria ballenera se desarrollaron y evolucionaron con el objetivo de maximizar la riqueza económica de los balleneros. Este proceso fue impulsado por el aprendizaje social y la experiencia adquirida a lo largo del tiempo. Los balleneros compartían conocimientos y habilidades, mejorando así sus técnicas de captura y aumentando sus posibilidades de obtener beneficios económicos.

Además, la interdependencia y la cooperación desempeñaron un papel fundamental en el desarrollo de las normas. Los balleneros reconocieron que enfrentaban desafíos comunes y comprendieron la importancia de colaborar entre sí. Establecieron normas que promovían la colaboración, como compartir información sobre la ubicación de las ballenas, evitando la competencia interna y maximizando las oportunidades de captura. Asimismo, limitaron el número de barcos permitidos en una cacería para evitar la sobreexplotación y garantizar un reparto equitativo de las oportunidades y ganancias.

La regulación también desempeñó un papel crucial en el desarrollo de las normas en la industria ballenera. Los balleneros comprendieron la importancia de preservar las poblaciones de ballenas a largo plazo para asegurar la continuidad de su actividad económica. Establecieron normas que regulaban la caza, como límites en el número de ballenas que podían ser capturadas en una temporada, y prohibiciones específicas, como la caza de ballenas preñadas o jóvenes. Estas normas buscaban evitar la sobreexplotación y mantener una población sostenible de ballenas, lo que a su vez contribuía a maximizar la riqueza a largo plazo.
\end{flushleft}

\section{¿Cuál fue la evidencia empírica utilizada por los autores para respaldar su hipótesis? Describe brevemente los datos y el enfoque metodológico utilizado en el estudio.}

\begin{flushleft}
    El autor utiliza la enciclopedia de Moby-Dick para sustentar su artículo, ya que en este texto ofrece una descripción detallada y vívida de la caza de ballenas en ese período de la historia. Melville presenta meticulosamente los aspectos técnicos y prácticos de la caza de ballenas, incluyendo información sobre los contratos entre los balleneros.
\end{flushleft}

\section{A partir de los resultados presentados en el artículo, ¿qué implicaciones tiene esta investigación para nuestra comprensión de las normas sociales y su influencia en la toma de decisiones económicas?}

\begin{flushleft}
    las normas sociales y culturales pueden tener un impacto significativo en la toma de decisiones económicas. En particular, el estudio muestra cómo las normas de la industria ballenera, como la competencia entre los capitanes balleneros y la necesidad de mantener una reputación positiva, pueden influir en el comportamiento económico y afectar la rentabilidad a largo plazo de la industria.

    Esto tiene implicaciones importantes para nuestra comprensión de cómo las normas sociales pueden afectar el comportamiento económico y cómo los incentivos económicos pueden ser moldeados por factores culturales y sociales. Además, el estudio sugiere que es importante considerar tanto los factores económicos como los culturales al analizar cualquier industria o mercado.
\end{flushleft}

\section{¿Qué limitaciones metodológicas o posibles críticas podrían plantearse sobre el enfoque utilizado en el estudio y cómo podría abordarse o mejorarse en futuras investigaciones?}

\begin{flushleft}
    El texto plantea un desafío central relacionado con sus fuentes y limitaciones teóricas. En primer lugar, es problemático que se base únicamente en una fuente sin contrastarla con otros datos o escritos. Además, el enfoque del texto muestra un sesgo anglocéntrico, sin tener en cuenta que los sistemas sociales de origen anglosajón tienen una marcada preferencia por la norma social. Esto puede influir en la forma de organización de los balleneros, lo que a su vez restringe el análisis a una perspectiva limitada y anglosajona. Para mejorar la calidad y la amplitud del análisis, sería recomendable considerar múltiples fuentes y tener en cuenta la diversidad de enfoques culturales y sociales.
\end{flushleft}
\end{document}