\documentclass[11pt]{article}
\usepackage{UF_FRED_paper_style}
\usepackage{lipsum}
\onehalfspacing

\setlength{\droptitle}{-5em} %% Don't touch

\title{Actividad en clase: Modelos
}

\author{Augusto Rico\\
    \href{mailto:arico@unal.edu.co}{\texttt{arico@unal.edu.co}}}

\date{\today}

\begin{document}

{\setstretch{.8} %% Don't touch
\maketitle}


% --------------------
\section{Ejemplo 1}
% --------------------
\begin{flushleft}
    Este caso se caracteriza como un típico dilema del prisionero. Si ambas empresas deciden no gastar, podría ser lo mejor para ambas. Sin embargo, si esto ocurre, una de las empresas tendrá incentivos para gastar y ganar el contrato para alcanzar el equilibrio de Nash donde ambas empresas gastan. En este escenario, el contrato se sortea aleatoriamente, pero ambas empresas verán afectada su utilidad ($g_i$).
\end{flushleft}

\section{Ejemplo 2}
% --------------------
\begin{flushleft}
    El juego de este ejemplo está caracterizado como un juego de subasta de primer precio, en el que las empresas compiten por los derechos de propiedad de una nueva tecnología. Dependiendo del valor privado $v_i$ de cada empresa, se determina el equilibrio de Nash. Si $v_i \geq V$, el equilibrio será que ambas empresas pujen por el bien, y la puja más alta será la que gane la subasta. Si $v_i < V$, el equilibrio será que ambas empresas no pujen por el bien y no se presente la subasta.
\end{flushleft}

\section{Ejemplo 3}
% --------------------
\begin{flushleft}
    En este ejemplo, el agente del contrato va a derivar su función de utilidad respecto al esfuerzo y obtendrá que $e=e'b'(e)/2$. Esta será su condición de maximización. De esta forma, si el pago por esforzarse es mayor que el esfuerzo causado, el agente se esforzará; de lo contrario, no lo hará.
\end{flushleft}

\section{Ejemplo 4}
% --------------------
\begin{flushleft}
    Tal como se explica en el ejemplo, la decisión óptima estará determinada por el impacto de la tecnología. Si se trata de una tecnología de alto impacto, la empresa tendrá $EU(1)>EU(0)$, lo que significa que decidirá invertir. Por otro lado, si se trata de una tecnología de bajo impacto, la empresa tendrá $EU(1)<EU(0)$ y no invertirá.
\end{flushleft}

\end{document}