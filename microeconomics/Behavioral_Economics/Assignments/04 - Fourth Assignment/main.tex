\documentclass[11pt]{article}
\usepackage{UF_FRED_paper_style}
\onehalfspacing
\setlength{\droptitle}{-5em} %% Don't touch

\title{Fourth Assignment: \\\textbf{\textit{Understanding social preferences with simple tests}}}


\author{Augusto Rico\\% Name author
    \href{mailto:arico@unal.edu.co}{\texttt{arico@unal.edu.co}}
    }

\date{\today}

\begin{document}
\maketitle

% %%%%%%%%%%%%%%%%%%%%%%%%%%%%%%%%%%%%%%%%%%%%%%%%%%%%%%%%%%
% %%%%%%%%%%%%%%%%%%%%%%%%%%%%%%%%%%%%%%%%%%%%%%%%%%%%%%%%%%
% BODY OF THE DOCUMENT
% %%%%%%%%%%%%%%%%%%%%%%%%%%%%%%%%%%%%%%%%%%%%%%%%%%%%%%%%%%
% %%%%%%%%%%%%%%%%%%%%%%%%%%%%%%%%%%%%%%%%%%%%%%%%%%%%%%%%%%

\begin{flushleft}
    Gary Charness and Matthew Rabin argue that, in contrast to conventional theory, people do not behave as rational agents seeking Pareto optima regardless of anything else. Instead, they contend that there are more desirable allocations, even if they are Pareto optimal, that players may not wish for in any way. This is because they often choose actions that do not maximize their monetary income when they affect the income of others, due to the potential for envy generation.\\~\\

    The authors propose a series of simple tests to assess people's preferences in realistic social situations where they must consider both their interests and those of others. These experiments are based on game theory and explore concepts such as reciprocity, fairness, and cooperation, providing valuable information on how people make decisions in social contexts and how their preferences influence their behavior.\\~\\
    
    Understanding social preferences, the authors argue, is essential for modeling human behavior more realistically than orthodox theory, which could contribute to more effective policies. The analysis of the ultimatum game is particularly noteworthy, as it reveals that not any Pareto optimal allocation is desirable, and for an offer to be accepted, it must be fair to all.\\~\\
    
    After all, what is found in the paper is that people, in general, have a genuine interest in social well-being, something that neoclassical theory does not seem to be interested in enough, given that agents should only be concerned with their own utility in this theory.\\~\\
    
    This helps us understand, for example, the theories of \textit{Symbolic Violence} by \citet{Foulcault_1987} and \citet{Bourdieu_2018}. These authors show that whenever there is a power relationship, there will, in some way, be a form of resistance that seeks justice in relationships. This phenomenon can be observed, for example, in the agrarian issue in Colombia, a historical struggle in which, even if it is a Pareto optimum, there will always be a quest to improve the condition and not accept the existing one in favor of a more just one. Such preferences can be represented more appropriately than in neoclassical theory.\\~\\
    
    Like the authors, it is interesting to consider the evaluation of other types of games, such as price wars in unequal competitive conditions. This could help determine whether, in rational situations, prices decrease, affecting everyone, or if situations are generated where the well-being of competitors is not greatly affected.\\~\\
    
    This topic is widely applicable to any social interaction. For example, in situations where someone excels in a class, even though theoretically this does not affect others, disadvantaged individuals may retaliate due to their perception of injustice. This can lead to a decrease in class participation, as standing out could prove counterproductive to social well-being.
\end{flushleft}

% %%%%%%%%%%%%%%%%%%%%%%%%%%%%%%%%%%%%%%%%%%%%%%%%%%%%%%%%%%
% %%%%%%%%%%%%%%%%%%%%%%%%%%%%%%%%%%%%%%%%%%%%%%%%%%%%%%%%%%
% REFERENCES SECTION
% %%%%%%%%%%%%%%%%%%%%%%%%%%%%%%%%%%%%%%%%%%%%%%%%%%%%%%%%%%
% %%%%%%%%%%%%%%%%%%%%%%%%%%%%%%%%%%%%%%%%%%%%%%%%%%%%%%%%%%
\newpage
\medskip

\newpage
\nocite{*}
\bibliography{references}

\end{document}