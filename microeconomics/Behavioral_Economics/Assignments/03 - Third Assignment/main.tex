\documentclass[11pt]{article}
\usepackage{UF_FRED_paper_style}
\onehalfspacing
\setlength{\droptitle}{-5em} %% Don't touch

\title{Third Assignment: \\ \textbf{Advances in Prospect Theory:} \\ \textbf{\textit{Cumulative Representation of Uncertainty}}}


\author{Augusto Rico\\% Name author
    \href{mailto:arico@unal.edu.co}{\texttt{arico@unal.edu.co}}
    }

\date{\today}

\begin{document}
\maketitle

% %%%%%%%%%%%%%%%%%%%%%%%%%%%%%%%%%%%%%%%%%%%%%%%%%%%%%%%%%%
% %%%%%%%%%%%%%%%%%%%%%%%%%%%%%%%%%%%%%%%%%%%%%%%%%%%%%%%%%%
% BODY OF THE DOCUMENT
% %%%%%%%%%%%%%%%%%%%%%%%%%%%%%%%%%%%%%%%%%%%%%%%%%%%%%%%%%%
% %%%%%%%%%%%%%%%%%%%%%%%%%%%%%%%%%%%%%%%%%%%%%%%%%%%%%%%%%%

\begin{flushleft}
    The authors argued that prospect theory provides a better explanation of decision making under risk and uncertainty than expected utility theory. Their paper aimed to create a descriptive theory of choice explaining various choice-related phenomena, such as framing effects, reference dependence, and loss aversion. They introduced cumulative prospect theory to extend its applicability to situations with uncertainty and any number of possible outcomes.\\~\\

    To support their argument, the authors conducted experiments involving 25 graduate students over three sessions. Their findings showed that cumulative prospect theory offers a unified explanation for both risky and uncertain situations and can account for the fourfold pattern of risk attitudes. However, they acknowledged that decision weights might be influenced by presentation, spacing, and outcome levels.\\~\\
    
    The experiment, conducted on a computer, presented participants with perspectives containing probabilities and monetary values. Participants had to indicate their preference between these perspectives and certain secure outcomes. To obtain a more precise estimate of what they considered secure, new sets of secure outcomes with adjusted values were displayed. The results were based on participants' observed choices, and the computer monitored the consistency of their responses. The analysis focused on perspectives with two monetary value outcomes and numerical probabilities, with 28 positive perspectives and 28 negative perspectives reported. Additionally, eight additional problems with slightly different approaches were also conducted.\\~\\
    
    Prospect theory outlines a fourfold pattern of risk attitudes, favoring risk aversion in gains and risk-seeking in losses, influenced by value and weighting functions. This pattern is widely observed, though individual differences exist. Empirical evidence supports this pattern in various situations, endorsing prospect theory's relevance in decision-making.\\~\\
    
    Key findings highlight cumulative prospect theory's ability to evaluate gains and losses comprehensively and explain the fourfold risk attitude pattern. However, there are acknowledged limitations, such as decision weights being influenced by presentation and the number of possible outcomes.\\~\\
    
    The paper proposes an interesting theory. Given that probabilities can lead us to act non-rationally, our mental calculations can deceive us. Additionally, in rational expected value calculations, it is not taken into account that sometimes we prefer a probability of not losing to a certainty of losing, even if the expected value of the former is lower than the latter. This is something important that is exposed in the paper. Similarly, it is shown how, in the case of gains, we often prefer secure gains rather than risking winning less. Therefore, rational theory may have difficulties explaining this.\\~\\
    
    It should be noted that the experiment was conducted with individuals who must have a perfect understanding of expected value theories. Nonetheless, favorable results were obtained, demonstrating that unlike other theories, even if one is aware that the other option is better mathematically, one will choose the option they perceive as better.\\~\\
    
    This happens often, especially when I play poker. I often try to avoid risk, especially because in poker, I can't see other players' cards, which creates real uncertainty about the odds. Therefore, I often have to play safely, as explained in the article.
\end{flushleft}
% %%%%%%%%%%%%%%%%%%%%%%%%%%%%%%%%%%%%%%%%%%%%%%%%%%%%%%%%%%
% %%%%%%%%%%%%%%%%%%%%%%%%%%%%%%%%%%%%%%%%%%%%%%%%%%%%%%%%%%
% REFERENCES SECTION
% %%%%%%%%%%%%%%%%%%%%%%%%%%%%%%%%%%%%%%%%%%%%%%%%%%%%%%%%%%
% %%%%%%%%%%%%%%%%%%%%%%%%%%%%%%%%%%%%%%%%%%%%%%%%%%%%%%%%%%
\newpage
\medskip

\newpage

\end{document}