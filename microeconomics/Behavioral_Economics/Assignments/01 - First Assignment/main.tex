\documentclass[11pt]{article}
\usepackage{UF_FRED_paper_style}
\onehalfspacing
\setlength{\droptitle}{-5em} %% Don't touch

\title{\text{First Assignment}
}

\author{Augusto Rico\\% Name author
    \href{mailto:arico@unal.edu.co}{\texttt{arico@unal.edu.co}}
    }

\date{\today}

\begin{document}
\maketitle

% %%%%%%%%%%%%%%%%%%%%%%%%%%%%%%%%%%%%%%%%%%%%%%%%%%%%%%%%%%
% %%%%%%%%%%%%%%%%%%%%%%%%%%%%%%%%%%%%%%%%%%%%%%%%%%%%%%%%%%
% BODY OF THE DOCUMENT
% %%%%%%%%%%%%%%%%%%%%%%%%%%%%%%%%%%%%%%%%%%%%%%%%%%%%%%%%%%
% %%%%%%%%%%%%%%%%%%%%%%%%%%%%%%%%%%%%%%%%%%%%%%%%%%%%%%%%%%

\begin{flushleft}
    In the paper, the author introduces the concept of an issue rooted in intuitive reasoning, which they term "the law of small numbers." This issue revolves around individuals' tendency to assume that a sample, regardless of its size, will adhere to the principles of the law of large numbers, thus reflecting the characteristics of the entire population.\\~\\

    To illustrate this, the author provides familiar examples such as gambling games. They highlight the misconception held by gamblers who fail to grasp that their own participation, regardless of its extent, represents an insignificantly small portion of the overall sample. Despite this reality, these gamblers consistently overestimate their statistical proficiency.\\~\\
    
    Delving deeper, the author emphasizes that this phenomenon occurs because the law of large numbers tends to manifest within these modest sample sizes, thereby increasing the likelihood of extreme cases materializing.\\~\\
    
    Rather than relying on experiments at this stage, the author chooses to present theoretical examples that are widely familiar. Furthermore, they reference instances of inaccuracies in cited literature, underscoring how even esteemed scholars fall victim to this bias. Remarkably, this issue becomes prominent enough to find its way into published works without any peer reviewer identifying the underlying statistical bias.\\~\\
    
    Moreover, this paper is helpful because it exemplifies the ideas discussed by Kant when arguing with Hume. Their debate centered on how people tend to seek reasons for things based on their existing beliefs, even when genuine reasons might be absent. In this context, individuals apply the concept of larger groups to various quantities, prompting the need to investigate whether, as Kant suggested, the problem indeed stems from our cognitive limitations, and whether individuals with greater expertise can resolve it. Consequently, it would be intriguing to assess this by involving different groups of people – some with average familiarity with statistics and others possessing significant knowledge in the field. These participants would encounter novel situations where the challenges of small number reasoning arise. This test could determine if those with advanced statistical knowledge still struggle to navigate these scenarios.\\~\\
    
    And the thing is, for instance, even individuals like me, who do not consider themselves statistical experts, encounter inference problems involving probabilities. However, a key method for rectifying such issues is by examining and comprehending the actual likelihood of events occurring. I have personally faced such situations numerous times, particularly in contexts related to casinos, akin to the author's examples.\\~\\
\end{flushleft}
% %%%%%%%%%%%%%%%%%%%%%%%%%%%%%%%%%%%%%%%%%%%%%%%%%%%%%%%%%%
% %%%%%%%%%%%%%%%%%%%%%%%%%%%%%%%%%%%%%%%%%%%%%%%%%%%%%%%%%%
% REFERENCES SECTION
% %%%%%%%%%%%%%%%%%%%%%%%%%%%%%%%%%%%%%%%%%%%%%%%%%%%%%%%%%%
% %%%%%%%%%%%%%%%%%%%%%%%%%%%%%%%%%%%%%%%%%%%%%%%%%%%%%%%%%%
\newpage
\medskip

\nocite{*}
\bibliography{references.bib} 

\newpage

\end{document}