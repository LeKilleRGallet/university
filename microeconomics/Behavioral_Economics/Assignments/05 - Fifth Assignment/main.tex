\documentclass[11pt]{article}
\usepackage{UF_FRED_paper_style}
\onehalfspacing
\setlength{\droptitle}{-5em} %% Don't touch

\title{Fifth Assignment: \\\textbf{\textit{Are people conditionally cooperative? Evidence from a public goods experiment}}}


\author{Augusto Rico\\% Name author
    \href{mailto:arico@unal.edu.co}{\texttt{arico@unal.edu.co}}
    }

\date{\today}

\begin{document}
\maketitle

% %%%%%%%%%%%%%%%%%%%%%%%%%%%%%%%%%%%%%%%%%%%%%%%%%%%%%%%%%%
% %%%%%%%%%%%%%%%%%%%%%%%%%%%%%%%%%%%%%%%%%%%%%%%%%%%%%%%%%%
% BODY OF THE DOCUMENT
% %%%%%%%%%%%%%%%%%%%%%%%%%%%%%%%%%%%%%%%%%%%%%%%%%%%%%%%%%%
% %%%%%%%%%%%%%%%%%%%%%%%%%%%%%%%%%%%%%%%%%%%%%%%%%%%%%%%%%%

\begin{flushleft}
    Neoclassical theory faces challenges in explaining public goods. The premise of rational individuals suggests the impossibility of generating public goods without specific mechanisms, challenging the essence of liberal theories by potentially allowing state intervention. However, reality reveals instances where public goods emerge from individuality, challenging the notion of \textit{homo œconomicus} by acknowledging cognitive illusions and biases in human behavior, leading to cooperation despite apparent \textit{individual irrationality}.\\~\\

    In a study by \citet{fischbacher2001people}, it was explored whether people would adopt the role of \textit{conditional cooperators}, contributing to the public good conditioned by the actions of others, as opposed to the \textit{free-riders} described in neoclassical theory. This analysis identifies motivations such as altruism, reciprocity, and aversion to inequity as drivers of this behavior.\\~\\

    This experiment models a strategic game where players, when interacting with others, are aware that they will not play with those same participants in the future. This strategy aims to avoid the consideration of a repeated game. According to the theory, in this dynamic, achieving the public good is less challenging, as players can penalize the \textit{free-riders} in future instances.\\~\\

    To achieve this, each player was assigned a number of chips, and groups were randomly formed for a single interaction, without considering long-term implications. Players were asked to make two types of decisions regarding the public good: one of 'unconditional contribution' and the other of a 'contribution table'. The first implied determining how many chips to invest in the project, while the second required specifying their level of contribution for different averages of contributions from others.\\~\\

    The results challenge economic theory by revealing that not everyone adopted the role of \textit{free-riders}; many increased their contributions as others did. However, on average, they did not contribute proportionally, behaving similarly to \textit{free-riders}. This suggests that people are \textit{conditional cooperators}, though not perfectly, challenging the ability of economic theory to explain these behaviors by ignoring human cognitive illusions.\\~\\

    However, it would be interesting to observe the decisions people would make if, unlike in this case where they are clear about the contributions others make and therefore their expected utilities, unlike in neoclassical theory where, a priori, one has no information about the action they are going to take, and therefore one ends up deciding not to contribute anything to the public good, remembering that everyone contributing is commonly a Pareto optimum.\\~\\

    It would also be relevant to analyze cases in which contributions to the public good, such as charity, do not directly impact individual utility. For instance, many donations to other countries mainly provide the donor with the satisfaction of giving. Exploring whether conditional cooperation patterns persist in these cases would be crucial, as well as exposing them to different scenarios to determine if donations seek recognition or if, for example, they would remain at the same level if completely anonymous.\\~\\

    In my personal experience, I've observed that those who perform charitable acts often wish to be acknowledged for it, seeking a form of validation rather than aiming for the selfless improvement of others, which would be morally correct.
\end{flushleft}

% %%%%%%%%%%%%%%%%%%%%%%%%%%%%%%%%%%%%%%%%%%%%%%%%%%%%%%%%%%
% %%%%%%%%%%%%%%%%%%%%%%%%%%%%%%%%%%%%%%%%%%%%%%%%%%%%%%%%%%
% REFERENCES SECTION
% %%%%%%%%%%%%%%%%%%%%%%%%%%%%%%%%%%%%%%%%%%%%%%%%%%%%%%%%%%
% %%%%%%%%%%%%%%%%%%%%%%%%%%%%%%%%%%%%%%%%%%%%%%%%%%%%%%%%%%
\newpage
\medskip

\newpage
\nocite{*}
\bibliography{references}

\end{document}