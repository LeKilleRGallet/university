\documentclass[12pt]{article}
\usepackage{UF_FRED_paper_style}
\usepackage{lipsum}
\onehalfspacing
\setlength{\droptitle}{-5em} %% Don't touch

\title{\textit{Working Paper:} \\ %% <-- THIS PART IS IMPORTANT!
Unraveling the Dynamics of the Colombian Health System: A Behavioral Analysis %% <-- THIS PART IS IMPORTANT!
}

\author{Augusto Rico\thanks{Thanks to Simon for guiding me in the \textit{Bourdieu's} analytical theory.}\\
    \href{mailto:arico@unal.edu.co}{\texttt{arico@unal.edu.co}}
% \and Second Author\\% Name author
%     \href{mailto:secondauthor@ufl.edu}{\texttt{secondauthor@ufl.edu}} %% Email author 2
    }

\date{\today}

\begin{document}

{\setstretch{.8} %% Don't touch
\maketitle
\begin{abstract}
\noindent\lipsum[2] %% Dummy text for the abstract. Erase before use.
% END CONTENT ABS------------------------------------------
~\\
\textit{\textbf{Keywords: }%
medical-care; habitus; tutela; key4.} \\ %% <-- Keywords HERE!
\textit{\textbf{JEL Classification: }%
D91; C7; I18.} %% <-- JEL code HERE!
\end{abstract}}



% %%%%%%%%%%%%%%%%%%%%%%%%%%%%%%%%%%%%%%%%%%%%%%%%%%%%%%%%%%
% %%%%%%%%%%%%%%%%%%%%%%%%%%%%%%%%%%%%%%%%%%%%%%%%%%%%%%%%%%
% BODY OF THE DOCUMENT
% %%%%%%%%%%%%%%%%%%%%%%%%%%%%%%%%%%%%%%%%%%%%%%%%%%%%%%%%%%
% %%%%%%%%%%%%%%%%%%%%%%%%%%%%%%%%%%%%%%%%%%%%%%%%%%%%%%%%%%

% --------------------
\section*{Introduction}
% --------------------
\begin{flushleft}
    Unlike other areas of economic study, health economics has a clear origin with \citet{arrow1963uncertainty}, who highlighted the issue of medical-care insurance, particularly information problems. Building upon Arrow's foundational work, this research delves into the dynamics of the Colombian health system, specifically examining the interplay among the Health Promoting Entities (\textit{EPS}), patients, and the government instituions. The unique focus on these interactions aims to provide insights into the incentives and behaviors that shape the functioning of the health sector in Colombia. By unraveling the complexities within this system, we contribute to the broader field of health economics, shedding light on the intricacies of healthcare delivery and decision-making in a context marked by \textit{information challenges}, \textit{moral hazard}, cognitive biases and social biases like \textit{Bourdieu's} habitus\\~\\

    While Arrow primarily addressed the moral risk inherent in patients' decisions, our inquiry refines the scope to unravel the intricacies of moral hazard on the part of \textit{EPS}. This transition in focus offers a distinctive perspective on the dynamics at play, exploring how the actions and strategies of \textit{EPS} contribute to the broader landscape of healthcare in Colombia.\par~\par
    
    %%%%%%
    % \begin{snippet}
    %     \begin{itemize}
    %         \item This research extends beyond theoretical exploration, seeking practical implications for policymakers and stakeholders involved in the Colombian health system.
    %         \item In this next section, we delve into the specific dimensions of moral hazard arising from the behavior of \textit{EPS}, providing a comprehensive analysis that contributes to a deeper understanding of the ethical contours within the Colombian health sector.
    %     \end{itemize}
    % \end{snippet}

\end{flushleft}

\section{Colombian Health System}

\begin{flushleft}

    In contrast to the American healthcare system, characterized by individually contracted private insurance, Colombia employs a variation of the \textbf{Bismarck model}. In this model, the state aims to provide coverage to all residents by paying a flat insurance fee per user to authorized companies known as \textit{EPS}. This fee remains identical for all citizens. Consequently, these authorized companies commit to serving these users and covering their medical treatments when necessary. This approach can be considered highly positive, given that the personal contribution to the common fund is determined by each individual's income. As outlined by \citet{perez2015mirada}, this system contributes to Colombia being among the countries with the lowest \textit{out-of-pocket} expenses for its citizens.\\~\\

    However, unlike private insurance systems such as the American one, in the Colombian system, users cannot easily switch between authorized companies. In contrast, given that health is a fundamental right in the country, users who believe that the \textit{EPS} is not providing them with the services for which they are paying can file a \textit{\textbf{tutela}}. Through this mechanism, users can ask a judge to compel the \textit{EPS} to provide the service. One advantage of this approach is that users who need to file a \textit{tutela} don't necessarily require a lawyer or legal knowledge.\footnote{According to the \textit{Defensoria del Pueblo} in their annual report on the \textit{\textbf{tutela}}, more than 80\% of tutelas are filed without a lawyer}  This implies that all users who feel they are not receiving the services they deserve should consider filing a claim, Since many times services are denied, which can even pose a threat to life, as evidenced by \citet{sanchez2014barreras} in their study where they show that \textit{EPS} can delay, on average, 20 days beyond the recommended period for breast cancer treatments.\\~\\
    
    % \begin{snippet}
    %     \begin{itemize}
    %         \item la dominacion esta inscrita atraves del habitus en el cuerpo como habitus (no tiene incentivos para la violencia sino que la violencia es necesaria por si misma)
    %         \item La violencia simbolica, mas que la violencia fisica o cualquier otra forma de coaccion mecanica, constituye el mecanismo principal de la reproduccion social, el medio mas potente del mantenimiento del orden. Bourdieu observa que el nucleo de la violencia simbolica se encuentra en la "doble naturalizacion" que es la consecuencia de la "inscripcion de lo social en las cosas y en el cuerpo". e.g la violencia simbolica solo es efectiva y constituye su poder unicamente cuando el oprimido acepta la doble naturalizacion de la misma o sea, cuando este no persive que hay violencia sino que es una situacion natural y por ende debe ser aceptada
    %         \item Otro campo donde Bourdieu ha estudiado los mecanismos de la violencia simbolica es el sistema de enseñanza. Este no se le presenta como un lugar donde se transmiten conocimientos de manera neutra sino un ambito donde se impone la cultura socialmente legitima: "Toda accion pedagogica es objetivamente una violencia simbolica en tanto que imposicion, por un poder arbitrario, de un arbitrario cultural" e.g el estado al omitir la instrumentalizacion contra violenta en contra del ciudadano esta permitiendo la violencia contra el mismo y asi mismo perdiendo poder frente a las eps (las eps captan los beneficios del monopolio de la violencia que debe ser unicamente del estado)
    %         \item Los esquemas mentales y culturales que funcionan como una matriz simbolica de la practica social se convierten en el verdadero fundamento de una teoria de la dominacion y de la politica: «de todas las formas de «persuasion clandestina», la mas implacable es la que se ejerce simplemente por el orden de las cosas»
    %         \item Esta es una lucha politica, cuyo objetivo principal es el estado, en particular su dimension simbolica, pues es la institucion que "detenta el monopolio de la violencia simbolica legitima": P. Bourdieu, Meditations pascaliennes, op. cit., p.222.
    %         \item la violencia se ejerce sobre todos los cuerpos a traves del habitus, pero no se naturaliza ni se adscribe igual a todos los cuerpos
    %         \item existe una super estructura que es la violencia simbolica, con una serie de implicaciones practicas (negar servicios y la gente piensa que gana dado que no percibe perdidas), que son generas por la ausencia de guia que deberia ser el estado tal como explica kant que deberia ser
    %         \item “La violencia simbolica es esa coercion que se instituye por mediacion de una adhesion que el dominado no puede evitar otorgar al dominante (y, por lo tanto, a la dominacion) cuando solo dispone para pensarlo y pensarse o, mejor aun, para pensar su relacion con el, de instrumentos de conocimiento que comparte con el y que, al no ser mas que la forma incorporada de la estructura de la relacion de dominacion, hacen que esta se presente como natural...”: BOURDIEU, Pierre, Meditaciones Pascalianas, Ed. Anagrama, 1999. Pag. 224/225.
    %         \item habitus como sistema de disposiciones adquiridas por los agentes sociales, como estructura estructurada estructurante, como sentido practico.
    %         \item las EPS en colombia no tienen incentivos para prestar correctamente su servicio, pues el cambiar de eps no es un trabajo facil, por lo que los usuarios deben acceder a una situacion de dominacion. dado esto, si las eps tuvieran una norma de comportamiento moral imperativamente categorica estas estarian perdiendo dinero.
    %     \end{itemize}
    % \end{snippet}
\end{flushleft}

\section{The Rational Model}

\begin{flushleft}

    In the rational model, we can assume a sequential ultimatum game, where the \textit{EPS} is the proposer, and the user accepts or rejects the proposal, with the difference that when the user rejects the proposal, an automatic payment is not obtained. Instead, we can assume that filing a \textit{\textbf{tutela}} is a state of nature with probability $\theta$, representing the historical percentage of health \textit{\textbf{tutelas}} won by users. In this game, the decisions unfold as follows: initially, the \textit{EPS} offers the user a service of $x\in(0,1)$, where the \textit{EPS}'s benefits are determined by the monetary difference between the offered service and the service needed given the medical conditions, $\pi=1-x$. Faced with this offer of $x$, the user has two options: accept what the \textit{EPS} offers and hence receive the payment of $x-1$\footnote{We must assume as a baseline that, given that this is a fundamental service that can even impact life, the user will spend their money to supplement what the \textit{EPS} does not provide.} from the \textit{EPS}, or reject the offer and place a tutela. In the latter case, the user may receive an additional payment of $t\in(x,1)$, causing a reduction in the \textit{EPS}'s benefit\footnote{For simplicity, we assume no legal representation costs}, which would now be $\pi=1-x-t$ if the court accepts the tutela. Payments would be equivalent to accepting the offer if the court rejects the \textit{\textbf{tutela}}, as the user receives no additional payment, and therefore, the \textit{EPS} pays nothing.\\~\\

    In this game with individual and rational agents, it is evident that both agents will always choose to reject the offer initially, as the user can obtain a higher expected payment than if they accepted the \textit{EPS} offer for any level of $\theta$. Consequently, the best response the \textit{EPS} can give is always to offer the minimum possible service ($x=0$). Offering a higher payment would reduce its expected benefits, and therefore, there would be no incentive to provide a better service unless compelled by a court.\\~\\

    It is possible to model this game as one with asimetric information where, given the user's medical ignorance, they are not sufficiently aware of the level of service they need. Nevertheless, the outcome will be the same since the user continues to have incentives to reject the offer and resort to legal stays, Even if some costs are assumed by the user when filing a \textit{tutela} (e.g., legal representation costs or waiting costs), we would end up with the scenario where the \textit{tutela} would be used if and only if $\theta t + \theta > \mathcal{C}$ is satisfied. This situation would occur in most cases, considering that \textit{EPS} best-response would be $x=0$ and then $t=1$ which means that $\theta > \mathcal{C}/2$, This implies that even assuming legal representation costs or other associated costs, filing \textit{tutelas} is a viable probability in the vast majority of cases. 
    
\end{flushleft}

\section{The Behavioral Model}

\begin{flushleft}

    Despite the previously proposed rational model, this is far from what actually happens. Even if we assume all possible costs and other negotiation barriers, we would end up with the same result using rationality, where the user would always decide to reject the health insurance provider's offer and resort to legal action, and the EPS would always provide the minimum possible service. This deviates from reality, as evidenced, for instance, by the \textit{Defensoría del Pueblo}, which only files legal actions in $0.71\%$ of all services provided. Similarly, the health insurance providers (\textit{EPS}) do not offer the minimum possible service, indicating that there may not be purely rational behavior on either side. Therefore, we will theorize about the possible reasons underlying this behavior.\\~\\

    We can start by Analyzing user behavior through the lens of prospect theory, as explained by \citet{thaler1980toward} and \citet{tversky1991loss}, we assume that users may encounter a Prospect Theory problem. They might contemplate initiating a \textit{tutela}, but given the associated costs, and considering that losing would entail not only bearing the \textit{tutela} costs but also dealing with medical service payments, akin to the \textit{loss aversion} phenomenon, users would feel this loss intensely if they don't prevail in the \textit{tutela}. Even with a minimal probability of losing, users might perceive a potential worsening of the \textit{Default Option} scenario, indicating susceptibility to a \textit{Risk Aversion} problem as elucidated by \citet{rabin2000risk}. They might prefer the "safe" option over a riskier one, even if the latter has a considerably higher expected payout.\\~\\

    Considering that, according to the \textit{Defensoria del Pueblo}, over $80\%$ of health-related legal actions result in victory, making it one of the categories with the highest probability of success, irrespective of educational level. Yet, many individuals, as highlighted by the \textit{Defensoria del Pueblo}, perceive filing a \textit{tutela} as requiring profound legal knowledge. Despite being designed for use by individuals with low literacy, there exists a prevalent belief that filing a \textit{tutela} would likely result in a loss\footnote{Personal experience: I've encountered instances where individuals, facing health service denials, particularly for high-risk diseases like cancer, from \textit{EPS}, expressed reluctance to pursue a \textit{tutela} due to their lack of legal connections and belief in their inability to navigate the process, even when they possess university degrees.}, reflecting an \textit{underconfidence} bias as explained by \citet{bjorkman1993realism}. Consequently, they are inclined to favor accepting the \textit{EPS} offer and privately paying for the denied service.\\~\\
    
    This inclination arises from the fact that, in the event of losing the \textit{tutela}, they would incur the same costs. Given their loss aversion tendencies, if they believe they can afford the denied service, they are likely to prefer paying for it instead of incurring other costs, even though the rational decision would be to reject the \textit{EPS} offer and pursue a \textit{tutela}.\\~\\

    Equally important to note within user behavior is the \textit{Commitment} that users who pay for social security undertake. Through this commitment, users engage in cooperative decision-making, contributing to the accessibility of health care for individuals without incomes. This is despite the rational decision in the short term, which may involve not paying, especially when in good health. It is crucial to remember that, in the case of unforeseen high-cost health problems (e.g., traffic accidents), users would still receive attention. In Colombia, health is a fundamental right for everyone.\\~\\

    Interestingly, despite users being the ones paying for the possibility of being attended to by the \textit{EPS} in Colombia, there are no instances of \textit{sunk cost fallacy} problems, as theoretically asserted by \citet{arrow1963uncertainty} and empirically observed by \citet{braverman2012assessment} in America. In the United States, users tend to use health services more than necessary, even when not needed. In contrast, Colombia maintains equivalent rates of health service usage between those who pay and those subsidized.\\~\\

    This can be explained by the Colombian model where workers contribute from their salaries. However, these contributors are often unaware of the extent of their contributions, leading users to not consider these costs when deciding whether to file a \textit{tutela}. If users were conscious of these contributions, they would have an additional incentive to file \textit{tutela}, at least for those who pay, believing they are reclaiming their money if successful. This could potentially lead to an increased inclination among users to pursue \textit{tutela}, given that today the proportion of users who effectively resort to tutela is not even higher than the proportion of users who are subsidized.\footnote{Statistics taken from the annual review on \textit{tutela} conducted by the Ministry of Health.}\\~\\

    To date, our focus has been solely on the user and their irrational behavior. Nevertheless, as mentioned earlier, \textit{EPS} also manifest behavior distinct from that anticipated by the rational model. Evidently, even though they do not provide the required service in its entirety, in the vast majority of cases, they fall considerably short of the minimum payment suggested by rationality. This hints at their failure to maximize profits as expected of a rational agent. Hence, a comprehensive examination of \textit{EPS} behavior becomes imperative.\\~\\

This divergence in behavior could potentially be attributed to an implicit \textit{Commitment} wherein a form of \textit{distributive fairness} is established. Under this understanding, the user agrees to the \textit{EPS} not furnishing the required service, provided that the service offered maintains a certain level of \textit{Fairness}. In reciprocation, the user refrains from initiating legal actions, thereby generating secure surpluses to sustain the \textit{EPS} over time. This phenomenon is discernible in historical data. When an \textit{EPS} encounters challenges such as bankruptcy or state intervention and consequently begins denying more services than historically denied, there is a noteworthy upswing in the number of \textit{tutelas} filed against it. This underscores users' aversion to inequity, indicating their capacity to tolerate service denials to some extent, contingent on them not being unduly excessive.

\end{flushleft}

\section{conclusion}
\newpage

% %%%%%%%%%%%%%%%%%%%%%%%%%%%%%%%%%%%%%%%%%%%%%%%%%%%%%%%%%%
% %%%%%%%%%%%%%%%%%%%%%%%%%%%%%%%%%%%%%%%%%%%%%%%%%%%%%%%%%%
% REFERENCES SECTION
% %%%%%%%%%%%%%%%%%%%%%%%%%%%%%%%%%%%%%%%%%%%%%%%%%%%%%%%%%%
% %%%%%%%%%%%%%%%%%%%%%%%%%%%%%%%%%%%%%%%%%%%%%%%%%%%%%%%%%%
\medskip

\nocite{*}
\printbibliography
\end{document}