\documentclass[11pt]{article}
\usepackage{UF_FRED_paper_style}
\usepackage{lipsum}
\onehalfspacing
\setlength{\droptitle}{-5em} %% Don't touch

\title{\textit{Working Paper:} \\ %% <-- THIS PART IS IMPORTANT!
    Title of the paper %% <-- THIS PART IS IMPORTANT!
}

\author{Augusto Rico\thanks{Thanks to Simon for guiding me in the \textit{Bourdieu's} analytical theory.}\\
    \href{mailto:arico@unal.edu.co}{\texttt{arico@unal.edu.co}}
% \and Second Author\\% Name author
%     \href{mailto:secondauthor@ufl.edu}{\texttt{secondauthor@ufl.edu}} %% Email author 2
    }

\date{\today}

\begin{document}

{\setstretch{.8} %% Don't touch
\maketitle
\begin{abstract}
\noindent\lipsum[2] %% Dummy text for the abstract. Erase before use.
% END CONTENT ABS------------------------------------------
~\\
\textit{\textbf{Keywords: }%
medical-care; symbolic violence; key3; key4.} \\ %% <-- Keywords HERE!
\textit{\textbf{JEL Classification: }%
D91; C7; I18.} %% <-- JEL code HERE!
\end{abstract}}



% %%%%%%%%%%%%%%%%%%%%%%%%%%%%%%%%%%%%%%%%%%%%%%%%%%%%%%%%%%
% %%%%%%%%%%%%%%%%%%%%%%%%%%%%%%%%%%%%%%%%%%%%%%%%%%%%%%%%%%
% BODY OF THE DOCUMENT
% %%%%%%%%%%%%%%%%%%%%%%%%%%%%%%%%%%%%%%%%%%%%%%%%%%%%%%%%%%
% %%%%%%%%%%%%%%%%%%%%%%%%%%%%%%%%%%%%%%%%%%%%%%%%%%%%%%%%%%

% --------------------
\section*{Introduction}
% --------------------
\begin{flushleft}
    Unlike other areas of economic study, health economics has a clear origin with \citet{arrow1963uncertainty}, who highlighted the issue of medical-care insurance, particularly information problems. Building upon Arrow's foundational work, this research delves into the dynamics of the Colombian health system, specifically examining the interplay among the Health Promoting Entities (\textit{EPS}), patients, and the government instituions. The unique focus on these interactions aims to provide insights into the incentives and behaviors that shape the functioning of the health sector in Colombia. By unraveling the complexities within this system, we contribute to the broader field of health economics, shedding light on the intricacies of healthcare delivery and decision-making in a context marked by \textit{information challenges}, \textit{moral hazard} and \textit{asymetric power} representen in \textit{\textbf{symbolic violence}}\\~\\

    While Arrow primarily addressed the moral risk inherent in patients' decisions, our inquiry refines the scope to unravel the intricacies of moral hazard on the part of \textit{EPS}. This transition in focus offers a distinctive perspective on the dynamics at play, exploring how the actions and strategies of \textit{EPS} contribute to the broader landscape of healthcare in Colombia.\par~\par
    
    %%%%%%
    \begin{snippet}
        \begin{itemize}
            \item This research extends beyond theoretical exploration, seeking practical implications for policymakers and stakeholders involved in the Colombian health system.
            \item In this next section, we delve into the specific dimensions of moral hazard arising from the behavior of \textit{EPS}, providing a comprehensive analysis that contributes to a deeper understanding of the ethical contours within the Colombian health sector.
        \end{itemize}
    \end{snippet}

\end{flushleft}

\section{Colombian Health System}

\begin{flushleft}

    In contrast to the American healthcare system, characterized by individually contracted private insurance, Colombia employs a variation of the \textbf{Bismarck model}. In this model, the state aims to provide coverage to all residents by paying a flat insurance fee per user to authorized companies known as \textit{EPS}. This fee remains identical for all citizens. Consequently, these authorized companies commit to serving these users and covering their medical treatments when necessary. This approach can be considered highly positive, given that the personal contribution to the common fund is determined by each individual's income. As outlined by \citet{perez2015mirada}, this system contributes to Colombia being among the countries with the lowest \textit{out-of-pocket} expenses for its citizens.\\~\\

    No obstante esto 


    \begin{snippet}
        \begin{itemize}
            \item la dominacion esta inscrita atraves del habitus en el cuerpo como habitus (no tiene incentivos para la violencia sino que la violencia es necesaria por si misma)
            \item La violencia simbolica, mas que la violencia fisica o cualquier otra forma de coaccion mecanica, constituye el mecanismo principal de la reproduccion social, el medio mas potente del mantenimiento del orden. Bourdieu observa que el nucleo de la violencia simbolica se encuentra en la "doble naturalizacion" que es la consecuencia de la "inscripcion de lo social en las cosas y en el cuerpo". e.g la violencia simbolica solo es efectiva y constituye su poder unicamente cuando el oprimido acepta la doble naturalizacion de la misma o sea, cuando este no persive que hay violencia sino que es una situacion natural y por ende debe ser aceptada
            \item Otro campo donde Bourdieu ha estudiado los mecanismos de la violencia simbolica es el sistema de enseñanza. Este no se le presenta como un lugar donde se transmiten conocimientos de manera neutra sino un ambito donde se impone la cultura socialmente legitima: "Toda accion pedagogica es objetivamente una violencia simbolica en tanto que imposicion, por un poder arbitrario, de un arbitrario cultural" e.g el estado al omitir la instrumentalizacion contra violenta en contra del ciudadano esta permitiendo la violencia contra el mismo y asi mismo perdiendo poder frente a las eps (las eps captan los beneficios del monopolio de la violencia que debe ser unicamente del estado)
            \item Los esquemas mentales y culturales que funcionan como una matriz simbolica de la practica social se convierten en el verdadero fundamento de una teoria de la dominacion y de la politica: «de todas las formas de «persuasion clandestina», la mas implacable es la que se ejerce simplemente por el orden de las cosas»
            \item Esta es una lucha politica, cuyo objetivo principal es el estado, en particular su dimension simbolica, pues es la institucion que "detenta el monopolio de la violencia simbolica legitima": P. Bourdieu, Meditations pascaliennes, op. cit., p.222.
            \item la violencia se ejerce sobre todos los cuerpos a traves del habitus, pero no se naturaliza ni se adscribe igual a todos los cuerpos
            \item existe una super estructura que es la violencia simbolica, con una serie de implicaciones practicas (negar servicios y la gente piensa que gana dado que no percibe perdidas), que son generas por la ausencia de guia que deberia ser el estado tal como explica kant que deberia ser
            \item “La violencia simbolica es esa coercion que se instituye por mediacion de una adhesion que el dominado no puede evitar otorgar al dominante (y, por lo tanto, a la dominacion) cuando solo dispone para pensarlo y pensarse o, mejor aun, para pensar su relacion con el, de instrumentos de conocimiento que comparte con el y que, al no ser mas que la forma incorporada de la estructura de la relacion de dominacion, hacen que esta se presente como natural...”: BOURDIEU, Pierre, Meditaciones Pascalianas, Ed. Anagrama, 1999. Pag. 224/225.
            \item habitus como sistema de disposiciones adquiridas por los agentes sociales, como estructura estructurada estructurante, como sentido practico.
            \item las EPS en colombia no tienen incentivos para prestar correctamente su servicio, pues el cambiar de eps no es un trabajo facil, por lo que los usuarios deben acceder a una situacion de dominacion. dado esto, si las eps tuvieran una norma de comportamiento moral imperativamente categorica estas estarian perdiendo dinero.
        \end{itemize}
    \end{snippet}
\end{flushleft}

\section{The rational model}

\begin{flushleft}
    \begin{snippet}
        \begin{itemize}
            \item
        \end{itemize}
    \end{snippet}
\end{flushleft}

\section{the behavioral model}

\begin{flushleft}
    \begin{snippet}
        \begin{itemize}
            \item
        \end{itemize}
    \end{snippet}
\end{flushleft}
\newpage

% %%%%%%%%%%%%%%%%%%%%%%%%%%%%%%%%%%%%%%%%%%%%%%%%%%%%%%%%%%
% %%%%%%%%%%%%%%%%%%%%%%%%%%%%%%%%%%%%%%%%%%%%%%%%%%%%%%%%%%
% REFERENCES SECTION
% %%%%%%%%%%%%%%%%%%%%%%%%%%%%%%%%%%%%%%%%%%%%%%%%%%%%%%%%%%
% %%%%%%%%%%%%%%%%%%%%%%%%%%%%%%%%%%%%%%%%%%%%%%%%%%%%%%%%%%
\medskip

\nocite{*}
\printbibliography
\end{document}