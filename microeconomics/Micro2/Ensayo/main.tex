\documentclass[11pt]{article}
\usepackage{UF_FRED_paper_style}
\usepackage{lipsum}
\onehalfspacing

\setlength{\droptitle}{-5em} %% Don't touch

\title{Reforma Agraria
}

\author{Augusto Rico\\
    \href{mailto:arico@unal.edu.co}{\texttt{arico@unal.edu.co}}
    }

\date{\today}

\begin{document}

{\setstretch{.8} %% Don't touch
\maketitle
\begin{abstract}
\lipsum[1] %% Dummy text for abstract. Erase before use.
% END CONTENT ABS------------------------------------------
\noindent
\textit{\textbf{Keywords: }%
key1; key2; key3; key4.} \\ %% <-- Keywords HERE!
\noindent
\textit{\textbf{JEL Classification: }%
Q12; C22; D81.} %% <-- JEL code HERE!

\end{abstract}


% %%%%%%%%%%%%%%%%%%%%%%%%%%%%%%%%%%%%%%%%%%%%%%%%%%%%%%%%%%
% %%%%%%%%%%%%%%%%%%%%%%%%%%%%%%%%%%%%%%%%%%%%%%%%%%%%%%%%%%
% BODY OF THE DOCUMENT
% %%%%%%%%%%%%%%%%%%%%%%%%%%%%%%%%%%%%%%%%%%%%%%%%%%%%%%%%%%
% %%%%%%%%%%%%%%%%%%%%%%%%%%%%%%%%%%%%%%%%%%%%%%%%%%%%%%%%%%

% --------------------
\section{Introduccion}
% --------------------
\begin{flushleft}
    La Tierra es considerado uno de los recursos fundamentales de la economia, que dada su caracteriztica escasa 
    y social ha sido centro de multiples discuciones economicas por parte de diversas escuelas,
    como han sido las discuciones del siglo \textbf{XIX} en Inglaterra sobre la nacionalizacion \citep{Ramos2007}.
    \\~\\
    La concentracion de la tierra, ha sido un problema ya que esto implica tambien una concentracion de la riqueza y el ingreso,
    afectando con ello el correcto funcionamiento de mercado y principalmente afectando los procesos de desarrollo en el pais,
    ya que estos buscaran unicamente su beneficio personal aprovechando la inexistencia de competencia perfecta para obtener mano de obra barata incluso en niveles inferiores a la subsistencia \citep{Cardenas1954}.
    hecho que impide el ahorro de la economia tal como plantea \citet{Say} y con ello el desarrollo capitalista, ya que se mantienen niveles de demanda artificialmente bajos,
    incapaces incluso de lograr demandar la incipiente produccion rudimentaria, es en este hecho en el que se enmarca la necesidad de una democratizacion de la propiedad de la tierra desde una perspectiva economica.
    \\~\\
    Por esto una \textit{reforma agraria} no se debe entender como un proceso de revolucion, sino en cambio como explica \citet{lenin_1971}
    se debe entender como una implantacion de logicas capitalistas en el sector agricola, ya que el fin ultimo de este proceso es un aumento 
    en productividad por la concepcion de firmas agricolas \citep{hamuy_1996}.
    \\~\\
    En el contexto colombiano la reforma agraria ha sido una necesidad imperante para el desarrollo y democratizacion de pais,
    dado que este es uno de los paises mas desiguales en terminos de tierra del continente e incluso del mundo,
    lo que ha impedido una profundizacion del capitalismo en el campo. No obstante, 
    dentro de Colombia misma existen casos como el del sector cafetero en la gran antioquia donde gracias a procesos de democratizacion se logro
    una penetracion del capitalismo y con ello una mejora del bienestar social\footnote{estas diferencias salariares pueden observarse en la \textit{\textbf{Mision de Empleo 2022}},
    particularmente en la figura 17 de \citep{AMM2021}}.
    
    \citep{Vergara2011}


\end{flushleft}


\newpage

% %%%%%%%%%%%%%%%%%%%%%%%%%%%%%%%%%%%%%%%%%%%%%%%%%%%%%%%%%%
% %%%%%%%%%%%%%%%%%%%%%%%%%%%%%%%%%%%%%%%%%%%%%%%%%%%%%%%%%%
% REFERENCES SECTION
% %%%%%%%%%%%%%%%%%%%%%%%%%%%%%%%%%%%%%%%%%%%%%%%%%%%%%%%%%%
% %%%%%%%%%%%%%%%%%%%%%%%%%%%%%%%%%%%%%%%%%%%%%%%%%%%%%%%%%%
\medskip

\bibliography{references.bib} 

\newpage

\end{document}