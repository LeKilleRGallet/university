\documentclass[11pt]{article}
\usepackage{UF_FRED_paper_style}
\usepackage{lipsum}
\onehalfspacing

\setlength{\droptitle}{-5em} %% Don't touch

\title{¿Pueden los mecanismos \textit{neoclásicos} de mercado  ayudarnos a lograr una redistribución justa de la tierra?
}

\author{Augusto Rico\thanks{Estudiante de pregrado en economia: Universidad Nacional de Colombia, Sede Bogota}\\
    \href{mailto:arico@unal.edu.co}{\texttt{arico@unal.edu.co}}
    }

\date{\today}

\begin{document}

{\setstretch{.8} %% Don't touch
\maketitle
\begin{abstract}
\begin{center}No, pero\dots\\~\\\end{center} %% Dummy text for abstract. Erase before use.
% END CONTENT ABS------------------------------------------
\noindent
\textit{\textbf{Keywords: }%
Eleccion Social; Reforma Agraria; Externalidades.} \\ %% <-- Keywords HERE!
\noindent
\textit{\textbf{JEL Classification: }%
D62	; D70; D30.}\\ %% <-- JEL code HERE!
\noindent
% \textit{\textbf{Temas: }%
% Asignación Justa; Maximización de bienestar social; Segundo Teorema del Bienestar; Reforma Agraria en Colombia.}


\end{abstract}



% %%%%%%%%%%%%%%%%%%%%%%%%%%%%%%%%%%%%%%%%%%%%%%%%%%%%%%%%%%
% %%%%%%%%%%%%%%%%%%%%%%%%%%%%%%%%%%%%%%%%%%%%%%%%%%%%%%%%%%
% BODY OF THE DOCUMENT 
% %%%%%%%%%%%%%%%%%%%%%%%%%%%%%%%%%%%%%%%%%%%%%%%%%%%%%%%%%%
% %%%%%%%%%%%%%%%%%%%%%%%%%%%%%%%%%%%%%%%%%%%%%%%%%%%%%%%%%%


% --------------------
\section{Introducción}
% --------------------
\begin{flushleft}
    La Tierra es considerado uno de los recursos fundamentales de la economía, que dada su característica escasa 
    y social ha sido centro de múltiples discusiones económicas por parte de diversas escuelas,
    como han sido las discusiones del siglo \textbf{XIX} en Inglaterra sobre la nacionalización \citep{Ramos2007}.
    \\~\\
    La concentración de la tierra, ha sido un problema, ya que esto implica también una concentración de la riqueza y el ingreso,
    afectando con ello el correcto funcionamiento de mercado y principalmente afectando los procesos de desarrollo en el país,
    puesto que estos buscaran únicamente su beneficio personal aprovechando la inexistencia de competencia perfecta para obtener mano de obra barata incluso en niveles inferiores a la subsistencia \citep{Cardenas1954}.
    Hecho que impide el ahorro de la economía, tal como plantea \citet{Say} y con ello el desarrollo capitalista, ya que se mantienen niveles de demanda artificialmente bajos,
    incapaces incluso de lograr demandar la incipiente producción rudimentaria, es en este hecho en el que se enmarca la necesidad de una democratización de la propiedad de la tierra desde una perspectiva económica.
    \\~\\
    Por esto, una \textit{reforma agraria} no se debe entender como un proceso de revolución, sino, en cambio, como explica \citet{lenin_1971}
    es mejor comprenderlo como una implantación de lógicas capitalistas en el sector agrícola, ya que el fin último de este proceso es un aumento 
    en productividad por la concepción de firmas agrícolas \citep{hamuy_1996}.
    \\~\\
    En el contexto colombiano, la reforma agraria ha sido una necesidad imperante para el desarrollo y democratización de país,
    dado que este es uno de los países más desiguales en términos de tierra del continente e incluso del mundo,
    lo que ha impedido una profundización del capitalismo en el campo. No obstante, 
    dentro de Colombia misma existen casos como el del sector cafetero en la \textit{Gran Antioquia}\footnote{Actuales departamentos de Antioquia, Caldas, Risaralda y Quindío} donde gracias a procesos de democratización se logró
    una penetración del capitalismo y con ello una mejora del bienestar social\footnote{se puede entender mejora en bienestar como mayor salario relativo, estas diferencias salariales pueden observarse en la \textit{\textbf{Misión de Empleo 2022}},
    particularmente en la figura 17 de \citep{AMM2021}}\citep{Vergara2011}. Y es que tal como expone \citet{Ocampo2001} en los años posteriores a reformas agrarias en Latinoamérica se observó un crecimiento significativo en el PIB agrario del respectivo país.\end{flushleft}

\section{Un modelo de economía agraria}

\begin{flushleft}
    A continuación se expondrá un modelo \textit{\textbf{teórico}} de una economía agraria en la cual se buscara entender si es posible
    que el mercado genere una redistribución parecida a una reforma agraria que logre maximizar el bienestar general de todos los individuos.
    \\~\\
    Para el mismo se utilizará el \textit{individualismo metodológico con agente representativo} con $n$ cantidad de agentes
    donde todos los agentes son productores y consumidores, y dispondrán de unas ciertas dotaciones.
    \\~\\
    El fin ultimó de este modelo es comprender si existe dotación inicial alguna que mediante los mecanismos del mercado derive
    en una \textit{asignación justa} o si, por el contrario, es necesario crear otros mecanismos para lograr una correcta redistribución de la tierra.\end{flushleft}

\subsection{Un modelo agrario sin capital}

\begin{flushleft}
    Para lograr una simplificación del modelo se asumirá que no existe capital, y por ende los únicos factores productivos
    serán únicamente \textit{tierra($L$)} con y \textit{trabajo($W$)} que producen un \textit{Bien($B$)} homogéneo y perfectamente divisible.
    Por lo que nuestra función de producción será: $B=L^\alpha W^{1-\alpha}|0<\alpha<1$ para asegurar que cumpla con las 
    características de una función de producción \textit{Cobb-Douglas}, de la misma forma los agentes tendrán una función
    de utilidad que será: $U(L,B)=BL$
\end{flushleft}

\subsection{Una teoría de la dominación mediante la propiedad de la tierra}

\begin{flushleft}
    El gran problema con la propiedad de la tierra, es que a diferencia de otros medios de producción o consumo, este en particular 
    es un foco de conflictos sociales debido a que la propiedad de la misma, después de todo, al rededor de la misma, existe un proceso
    de poder por parte del latifundista contra el desposeído \citep{overbeek_silva_1986}, ya que como afirma \citet{Moyano_Sevilla1978} más allá del proceso 
    productivo la tierra otorga estatus y poder al poseedor.
    \\~\\
    Por lo anterior es necesario hacer presente una externalidad en la utilidad de los agentes, que estará
    representada por la cantidad de tierra poseída relativa a la media de la población general, que representaremos matemáticamente como $L_i/\bar{L}$
    donde $\bar{L}$ es igual a la media de tierra poseída por todos los agentes \footnote{representado matemáticamente como $\sum L_i/n$}. Por lo anterior, nuestra nueva función de Utilidad
    será $U_i(L_i,B_i)=B_iL_i(L_i/\bar{L})$
\end{flushleft}

\subsection{Una economía del bienestar con \textit{axiomas éticos}}

\begin{flushleft}
    Por lo anteriormente explicado nos interesa una sociedad equitativa, por lo que al maximizar nuestro bienestar social vamos a presidir
    de utilizar una función de bienestar social tipo \textbf{Bentham-Mill}\footnote{esta función de bienestar es $\sum U_i$}, ya que en esta el bienestar social es maximizable con un latifundista 
    con todas las tierras de la sociedad, lo que nos puede generar problemas sociales. 
    \\~\\
    En cambio, utilizaremos una función de bienestar social tipo \textbf{Bernoulli-Nash}\footnote{esta función de bienestar es $\prod U_i$}
    tal como expone \citet{nash_1950} cumpliendo los axiomas éticos de \citet{nash_1953}, con lo que obtenemos que el bienestar social únicamente es maximizable
    cuando todos los agentes obtienen el mismo nivel de utilidad dado que por preceptos liberales se considera que todos los agentes tendrán igual ponderación.
    \\~\\
    por lo que tendremos que el bienestar se va a maximizar si y solo si $\prod U_i = (\bar{L}(\bar{L}^\alpha \bar{W}^{1-\alpha}))^n$, que como se podrá notar nos asegura también una \textit{asignación justa}
    para todos los agentes, ya que todos obtendrían una misma cantidad de tierra y bienes.
\end{flushleft}

\subsection{Dotaciones y asignaciones}

\begin{flushleft}
    En esta economía se asumirá que para todos los agentes su dotación de fuerza de trabajo es únicamente ellos mismos,
    por lo que $\omega_i^W=1$, ya que todos los agentes tienen iguales capacidades de trabajo, mientras que la dotación de tierra
    si es posible que sea distinta para cada agente, por lo que $\omega_i^L$ solo estará sujeta a encontrarse entre cero y
    la cantidad de tierra total en la economía, pero no a ser equitativa.
    \\~\\
    Dadas estas restricciones de dotaciones vamos a tener que mediante el intercambio será imposible llegar a la asignación justa
    para todo caso donde las dotaciones iniciales de la tierra sean iguales para todos los agentes, todo esto ocasionado principalmente 
    por la restricción en la dotación de trabajo, ya que para que la única pendiente de precios que es capaz de pasar por una dotación $\omega_i=(L_i,1)$ distinta a la asignación justa
    y la asignación justa debe estar caracterizada por tener que $P_L=0$ lo que es imposible con nuestra función de utilidad mientras se tenga agentes \textit{racionales},
    y esto igualmente nos ocasionaría un problema de no existencia de un equilibrio \textit{walrasiano}, ya que todos los agentes tendrían una demanda infinita de tierra.    \subsection{Una forma alternativa para obtener la asignacion justa: \textit{La Eleccion Social}}
    
    Por lo anterior, para llegar a la asignación justa se tendrá que recurrir a métodos de asignación distintos a los del mercado, esto enmarcado 
    por ejemplo en el pensamiento de \citet{arrow_1973} quien argumenta críticamente que aunque el mercado es un mecanismo típico para tomar decisiones económicas, 
    y el resultado de decisiones individuales reflejadas en el precio, es imposible resolver mediante este el conflicto entre los incentivos individuales y el bienestar colectivo,
    lo que nos lleva a que el mercado va a ser ineficiente para obtener el bienestar colectivo, y es que tal como afirma \citet{arrow_1951} existen dos métodos de elección social: 
    el \textit{mercado} y el \textit{voto}, el primero para decisiones económicas y políticas respectivamente, y dado que la reforma agraria aun si puede ser un beneficio para la 
    economía es una decisión puramente política en la que el mercado será insuficiente.
    \\~\\
    Por esto, para lograr la asignación justa se tendrá que recurrir a una \textit{función de bienestar social}, en nuestro caso una
    en la que los agentes en conjunto posterior al intercambio decidan sobre si hacer o no una reforma agraria. Para esto definiremos una
    \textit{función de bienestar social de la mayoría por pares}\footnote{véase la definición 8 de \citet{Lozano2022} para la simbología utilizada} con dos opciones posibles\footnote{dado que el número de opciones es menor a 3 no hay que preocuparse por el \textit{Teorema de imposibilidad de Arrow}}
    Hacer Reforma Agraria\textbf{(RA)} y No Hacer Reforma Agraria\textbf{(NRA)} con el siguiente sistema de decisión:

    \begin{center}

        $ 
     \begin{cases}
         NRA \succsim_i RA ~ \text{ si y solo si } ~  L_i/\bar{L} \le 1\\
         RA \succsim_i NRA ~ \text{ si y solo si } ~  L_i/\bar{L} \ge 1
     \end{cases}
         $
     \end{center}

     Mediante el cual, si obtenemos que $N^\rho(RA,NRA)>N^\rho(NRA,RA)$ entonces tendremos que $RA \succ^\varphi NRA$ y se procederá a hacer una reforma agraria
     con la cual se reasignaran las dotaciones de forma justa logrando el máximo bienestar social, caso contrario se mantendrán las asignaciones dadas por el mercado.
     \\~\\
     Sin olvidar que las elecciones son hechas posteriores al intercambio, vamos a tener un nuevo problema y es que en el momento anterior a la elección, en el que 
     los agentes caracterizados por $NRA \succsim_i RA$ gracias a su perfecta racionalidad prevendrán una perdida de utilidad por la reforma agraria y con ello van
     a tener nuevos incentivos, por ende podrían tomar la decisión de donar algo de su tierra para evitar la reforma enmarcada en la siguiente maximización\footnote{maximizacion basada en el famoso problema de algoritmica de la mochila.}:
     
    \begin{center}
        \begin{align*}
            & \boldsymbol{\max_{U_i} ~ U_i \left(L_i^*-\varepsilon,B_i^*\right)}\\
            \text{S.A.} \\ &~ \boldsymbol{N^\rho(NRA,RA)>N^\rho(RA,NRA)}\\
            & \boldsymbol{U_i \left(L_i^*-\varepsilon,B_i^*\right) > U_i \left(L^j,B^j\right)}\\
            & \boldsymbol{\varepsilon \ge} 0
        \end{align*}
    \end{center}
    
    Proceso de maximización que dado el mercado no se encuentre en la asignación justa y que los bienes neoclásicos son perfectamente divisibles
    la anterior maximización siempre tendrá solución al menos para un agente, agente que lograra evitar la reforma agraria, y por ende la
    Asignación justa y la maximización de bienestar social, pero maximizando su utilidad individual como la de sus donatarios al nuevo máximo posible.

\end{flushleft}

\section{Conclusiones}

\begin{flushleft}
    \begin{itemize}
        \item Tal como se mostró, mediante el mercado es prácticamente imposible lograr una reforma agraria en esta economía
        \item Tal como se mostró, mediante la democracia es prácticamente imposible lograr una reforma agraria en esta economía
        \item Dado que existen incentivos individuales para evitar una reforma agraria será complejo que sin un ente central
        que determina sin importarle las decisiones individuales y únicamente prioriza el bienestar social, se pueda lograr una reforma agraria en esta economía
        \item Aunque la democracia es insuficiente para lograr una reforma agraria tal como se expuso, si genera asignaciones más equitativas respecto el mercado y con ello mayor bienestar social.
    \end{itemize}
\end{flushleft}


\newpage

% %%%%%%%%%%%%%%%%%%%%%%%%%%%%%%%%%%%%%%%%%%%%%%%%%%%%%%%%%%
% %%%%%%%%%%%%%%%%%%%%%%%%%%%%%%%%%%%%%%%%%%%%%%%%%%%%%%%%%%
% REFERENCES SECTION
% %%%%%%%%%%%%%%%%%%%%%%%%%%%%%%%%%%%%%%%%%%%%%%%%%%%%%%%%%%
% %%%%%%%%%%%%%%%%%%%%%%%%%%%%%%%%%%%%%%%%%%%%%%%%%%%%%%%%%%
\medskip

\bibliography{references.bib} 

\newpage

\end{document}