\documentclass[11pt]{article}
\usepackage{UF_FRED_paper_style}
\usepackage{lipsum}
\onehalfspacing

\setlength{\droptitle}{-5em} %% Don't touch

\title{Teoria de Juegos - Parcial 2}

\author{Augusto Rico\\
    \href{mailto:arico@unal.edu.co}{\texttt{arico@unal.edu.co}}}

\date{\today}

\begin{document}

\setstretch{.8} %% Don't touch
\maketitle


% %%%%%%%%%%%%%%%%%%%%%%%%%%%%%%%%%%%%%%%%%%%%%%%%%%%%%%%%%%
% %%%%%%%%%%%%%%%%%%%%%%%%%%%%%%%%%%%%%%%%%%%%%%%%%%%%%%%%%%
% BODY OF THE DOCUMENT
% %%%%%%%%%%%%%%%%%%%%%%%%%%%%%%%%%%%%%%%%%%%%%%%%%%%%%%%%%%
% %%%%%%%%%%%%%%%%%%%%%%%%%%%%%%%%%%%%%%%%%%%%%%%%%%%%%%%%%%

% --------------------
\section{Primer grupo de ejercicios para el Parcial \#2}

% --------------------
\subsection{Ejercicio 1}
\subsubsection{Punto a}
\begin{flushleft}
    \begin{align*}
        U_1 &= p(q(2)+(1-q)(-1))+(1-p)(q(2)+(1-q)(6))\\
        U_1 &= p(2q-1+q)+(1-p)(2q+6-6q)\\
        U_1 &= p(3q-1)+(1-p)(-4q+6)\\
        U_1 &= 3pq-p+6-4q+4pq-6p\\
        U_1 &= 7pq-7p-4q+6\\
        U_1 &= p(7q-7)-4q+6\\
        \frac{\partial U_1}{\partial p} &= 7q-7=0\to q=7/7=1\\
    \end{align*}
    \begin{align*}
        U_2 &= q(p(-1)+(1-p)(5))+(1-q)(p(2)+(1-p)(1))\\
        U_2 &= q(-p-5p+5)+(1-q)(2p-p+1)\\
        U_2 &= q(-6p+5)+(1-q)(p+1)\\
        U_2 &= -6pq+5q+p-pq-q+1\\
        U_2 &= -7pq+4q+p+1\\
        U_2 &= q(-7p+4)+p+1\\
        \frac{\partial U_2}{\partial q} &= -7p+4=0\to p=4/7\\
    \end{align*}

    entonces tendremos que las correspondencias de mejor respuesta seran:\\~\\

    \begin{minipage}{0.3\textwidth}
        \[
        \begin{aligned}
            &Mr_1(q) = \begin{cases}
                0 &\text{si } q<1 \\
                x &\text{si } q=1
            \end{cases} \\
            \\
            &Mr_2(p) = \begin{cases}
                0 &\text{si } p>4/7 \\
                x &\text{si } p=4/7\\
                1 &\text{si } p<4/7
            \end{cases}
        \end{aligned}
        \]
    \end{minipage}
    \begin{minipage}{0.6\textwidth}
        \begin{center}
            \begin{tikzpicture}
                \begin{axis}[
                    xlabel={$p$},
                    ylabel={$q$},
                    xmin=0, xmax=1.1,
                    ymin=0, ymax=1.1,
                    xtick={0,4/7,1},
                    ytick={0,1},
                    xticklabels={0, $\frac{4}{7}$, 1},
                    yticklabels={0, 1},
                    axis lines=middle,
                    enlargelimits=true,
                    clip=false,
                    width=8cm,
                    height=6cm,
                    every x tick/.style={opacity=0},
                    every y tick/.style={opacity=0}
                ]
                    % Función Mr_1(q)
                    \addplot [domain=0:1,blue,thick] {1};
                    \addplot [domain=0:1,blue,thick] coordinates {(0, 0) (0, 1)};
                    
                    % Función Mr_2(p)
                    \addplot [domain=0:4/7,red,thick] {0.99};
                    \addplot [domain=4/7:1,red,thick] {0};
                    \addplot [domain=0:1,red,thick] coordinates {(4/7, 0) (4/7, 1)};
        
                    % equilibrios
                    %mixtos
                    \draw[green,thick] (axis cs:4/14,0.99) ellipse [x radius=4/13, y radius=0.1];
                    \node[anchor=north] (source) at (axis cs:4/14,0.4){\footnotesize \begin{tabular}{c}Infinitos\\ Equilibrios\end{tabular}};
                    \node (destination) at (axis cs:4/14,0.9){};
                    \draw[->](source)--(destination);
                \end{axis}
            \end{tikzpicture}
        \end{center}
    \end{minipage}
\end{flushleft}

\subsubsection{Punto b}

\begin{flushleft}
    \begin{align*}
        U_1 &= p(q(3)+(1-q)(4))+(1-p)(q(2)+(1-q)(10))\\
        U_1 &= p(3q+4-4q)+(1-p)(2q+10-10q)\\
        U_1 &= p(-q+4)+(1-p)(-8q+10)\\
        U_1 &= -pq+4p-8q+10+8pq-10p\\
        U_1 &= 7pq-6p-8q+10\\
        U_1 &= p(7q-6)-8q+10\\
        \frac{\partial U_1}{\partial p} &= 7q-6=0\to q=6/7\\
    \end{align*}
    \begin{align*}
        U_2 &= q(p(6)+(1-p)(11))+(1-q)(p(4)+(1-p)(8))\\
        U_2 &= q(6p+11-11p)+(1-q)(4p+8-8p)\\
        U_2 &= q(-5p+11)+(1-q)(-4p+8)\\
        U_2 &= -5pq+11q-4p+8+4pq-8q\\
        U_2 &= -pq+3q-4p+8\\
        U_2 &= q(-p+3)-4p+8\\
        \frac{\partial U_2}{\partial q} &= -p+3=0\to p=3\\
    \end{align*}

    entonces tendremos que las correspondencias de mejor respuesta seran:\\~\\

    \begin{minipage}{0.3\textwidth}
        \[
        \begin{aligned}
            &Mr_1(q) = \begin{cases}
                0 &\text{si } q<6/7 \\
                x &\text{si } q=6/7 \\
                1 &\text{si } q>6/7
            \end{cases} \\
            \\
            &Mr_2(p) = \begin{cases}
                0  &\text{si } p \leq 1
            \end{cases}
        \end{aligned}
        \]
    \end{minipage}
    \begin{minipage}{0.6\textwidth}
        \begin{center}
            \begin{tikzpicture}
                \begin{axis}[
                    xlabel={$p$},
                    ylabel={$q$},
                    xmin=0, xmax=1.1,
                    ymin=0, ymax=1.1,
                    xtick={0,1},
                    ytick={0,6/7,1},
                    xticklabels={0, 1},
                    yticklabels={0, $\frac{6}{7}$, 1},
                    axis lines=middle,
                    enlargelimits=true,
                    clip=false,
                    width=8cm,
                    height=6cm,
                    every x tick/.style={opacity=0},
                    every y tick/.style={opacity=0}
                ]
                    % Función Mr_1(q)
                    \addplot [domain=0:1,blue,thick] {6/7};
                    \addplot [domain=0:1,blue,thick] coordinates {(0, 0) (0, 6/7)};
                    \addplot [domain=0:1,blue,thick] coordinates {(1, 6/7) (1, 1)};
                    
                    % Función Mr_2(p)
                    \addplot [domain=0:1,red,thick] coordinates {(0, 0) (1, 0)};
        
                    % equilibrios
                    % puros
                    \node[anchor=north] (source) at (axis cs:-0.1,-0.1){\footnotesize \begin{tabular}{c}Eq de Nash\\ Puro $(x_1,y_1)$ \end{tabular}};
                    \node (destination) at (axis cs:0,0){};
                    \draw[->](source)--(destination);
                    \addplot[mark=*] coordinates {(0, 0)};
                \end{axis}
            \end{tikzpicture}
        \end{center}
    \end{minipage}
\end{flushleft}

\subsection{Punto 2}
\subsubsection{Punto a}
\begin{flushleft}
    \begin{align*}
        p(4)+q(9)&=p(6)+q(5)\to p=2q\\
        p(4)+q(9)&=p(7)+q(1) \to p=8q/3\\
    \end{align*}
    dado estos resultados sabemos que este sistema de ecuaciones no va a tener ecuaciones,
    por lo que como solucion debemos probar si asumiendo alguna de las estrategias $\{L,M,R\}$
    con probabilidad 0 podemos obtener un equilibrio realista.\\~\\
    
    \textbf{Caso 1: }$L=0$\\

    \begin{align*}
        q(5)+(1-q)(8)&=q(6)+(1-q)(5)\\
        5q+8-8q&=6q+5-5q\\
        -3q+8&=q+5\\
        -4q&=-3\\
        q&=3/4\\
    \end{align*}
    \begin{align*}
        p(6)+(1-p)(5)&=p(7)+(1-p)(1)\\
        6p+5-5p&=7p+1-p\\
        p+5&=6p+1\\
        -5p&=-4\\
        p&=4/5\\
    \end{align*}
    como estamos asumiendo que $L=0$ debemos obtener tambien que $E_M>E_L$ y $E_R>E_L$, caso contrario se debe desestimar este caso.\\
    \begin{align*}
        E_L&=(4/5)(4)+(1/5)(9)=5\\
        E_M&=(4/5)(6)+(1/5)(5)=5.8\\
        E_R&=(4/5)(7)+(1/5)(1)=5.8
    \end{align*}
    como $E_M,E_R>E_L$ podemos decir que este es un equilibrio de Nash.\\~\\
\end{flushleft}
\subsubsection{Punto b}
\begin{flushleft}
    \begin{align*}
        E_T=E_B, ~&~ E_T = E_U\\
        q_1(0)+q_2(5)+(1-q1-q2)(4)&=q_1(4)+q_2(0)+(1-q1-q2)(5)\\
        q_1(0)+q_2(5)+(1-q1-q2)(4)&= q_1(5)+q_2(4)+(1-q1-q2)(0)\\~\\
        5q_2+4-4q_1-4q_2&=4q_1+5-5q_1-5q_2\\
        q_2&=(3q_1+1)/6\\~\\
        5q_2+4-4q_1-4q_2&=5q_1+4q_2\\
        -3q_2&=9q_1-4\\
        -3\left(\frac{3q_1+1}{6}\right)&=9q_1-4\\
        -9q_1-3&=54q_1-24\\
        63q_1&=21\\~\\
        q_1&=21/63=1/3\\
        q_2&=(3(1/3)+1)/6=1/3\\
        q_3&=1-1/3-1/3=1/3\\~\\
        \text{por simetria sabemos que: }\\
        p_1=q_1, p_2&=q_2,p_3=q_3\\~\\
        \text{por lo que el unico equilibrio de este juego es: }\\
        ((1/3T,1/3B,1/3U),&(1/3L,1/3M,1/3R))
    \end{align*}
\end{flushleft}

\subsection{Punto 3}
\begin{flushleft}
    para este ejemplo tendremos que los jugadores $1,2,3$ juegan con las probabilidades
    $p,q,b$ respectivamente, y sabiendo que las estrategias y por ende estas probabilidades son
    independientes podemos hacer lo siguiente.\\
    \begin{align*}
        E_{x_1}= qb(0)+((1-q)b)(6)&+(q(1-b))(4)+((1-q)(1-b))(0)\\
        E_{y_1}= qb(5)+((1-q)b)(0)&+(q(1-b))(0)+((1-q)(1-b))(0)\\
        E_{x_2}= pb(0)+((1-p)b)(4)&+(p(1-b))(6)+((1-p)(1-b))(0)\\
        E_{y_2}= pb(5)+((1-p)b)(0)&+(p(1-b))(0)+((1-p)(1-b))(0)\\
        E_{x_3}= pq(0)+((1-p)q)(6)&+(p(1-q))(4)+((1-p)(1-q))(0)\\
        E_{y_3}= pq(5)+((1-p)q)(0)&+(p(1-q))(0)+((1-p)(1-q))(0)\\~\\
        \text{Simplificamos y resolvemos}&\text{ antes de igualar}\\~\\
        E_{x_1}&=4q+6b-10qb\\
        E_{y_1}&=5qb\\
        E_{x_2}&=6p+4b-10pb\\
        E_{y_2}&=5pb\\
        E_{x_3}&=4p+6q-10pq\\
        E_{y_3}&=5pq\\~\\
        \text{Igualamos}\\~\\
        4q+6b-10qb&=5qb\\
        6p+4b-10pb&=5pb\\
        4p+6q-10pq&=5pq\\~\\
        \text{Despejamos la primera }& \text{ecuacion}\\~\\
        6b-15qb&=-4q \to b=-\frac{4q}{6-15q}\\~\\
        \text{Sustituimos en la segunda }& \text{ecuacion}\\~\\
        6p+4\left(-\frac{4q}{6-15q}\right)-10p\left(-\frac{4q}{6-15q}\right)&=5p\left(-\frac{4q}{6-15q}\right)\\
        6p+\frac{20pq}{2-5q}&=\frac{16q}{6-15q}\\
        p=\frac{\frac{16q}{6-15q}}{6+\frac{20q}{2-5q}}&=\frac{8q}{18-15q}\\~\\
        \text{Sustituimos en la tercera }& \text{ecuacion}\\~\\
        4\left(\frac{8q}{18-15q}\right)+6q-10\left(\frac{8q}{18-15q}\right)q&=5\left(\frac{8q}{18-15q}\right)q\\
        \frac{32q}{18-15q}+6q-\frac{80q^2}{18-15q}&=\frac{40q^2}{18-15q}\\
        32q+6q(18-15q)-80q^2&=40q^2\\
        140q-210q^2=0\\
        q(140-210q)=0 &\to q_1=0\\
        q(140-210q)=0 &\to q_2=\frac{140}{210}=2/3\\~\\
        \text{remplazamos ambas opciones en}& \text{ las ecuaciones anteriores}\\~\\
        p_1=\frac{8(0)}{18-15(0)}&=0\\
        p_2=\frac{8(2/3)}{18-15(2/3)}&=2/3\\~\\
        b_1=-\frac{4(0)}{6-15(0)}&=0\\
        b_2=-\frac{4(2/3)}{6-15(2/3)}&=2/3\\~\\
    \end{align*}
    con esto obtenemos dos equilibrios, no obstante $(0,0,0)$ es un equilibrio puro,
    que corresponde a las estrategias puras $(x_1,x_2,x_3)$, por lo que al ser una estrategia
    degenerada, no se va a considerar como una estrategia mixta, por lo que el unico equilibrio
    mixto sera $(2/3,2/3,2/3)$, que junto con los tres equilibrios puros $((y_1,x_2,x_3),(x_1,y_2,x_3),(x_1,x_2,y_3))$
    nos da un total de 4 equilibrios de nash en el juego, una cantidad par.
\end{flushleft}

\subsection{Punto 4}

\begin{flushleft}
    Para esta demostracion se realizara una prueba por contradiccion, asumiendo entonces que
    $\exists\sigma^w|\sigma^w\in\Gamma^{w}\land\sigma^w\notin\Gamma$ donde $\Gamma \to_{w} \Gamma^{w}$
    si esto es cierto, entonces debe cumplirse que:
    \begin{align*}
        \varphi := \left\{\sigma_i'\in\Gamma|u_i(\sigma_i',\sigma_{-i})>u_i(\sigma^w,\sigma_{-i})\right\} \neq \emptyset
    \end{align*} 
    pero en esta definicion nace un gran problema, y es que al ser la dominancia debil
    un orden parcial estricticto, por lo que es una relacion irreflesiva y transitiva, y $\varphi$
    ser un conjunto finito, implica que la strategia $\sigma_i'$ no va a ser debilmente dominada en $\Gamma$
    por ninguna otra estrategia en $\varphi$, no obstante $\forall\sigma_i'\in\varphi$ es eliminada
    en la reduccion $\Gamma \to_{w} \Gamma^{w}$, donde tenemos que $\sigma^w$ es un equilibrio
    de $\Gamma^{w}$. por lo que debe entonces existir una estrategia $\sigma^*_i\in\Gamma$
    que domina debilmente a $\sigma_i'$ en $\Gamma$, lo que nos lleva entonces a que:
    \begin{align*}
        u_i(\sigma^*_i,\sigma_{-i})\geq u_i(\sigma_i',\sigma_{-i})
    \end{align*}
    lo que implica entonces que $\sigma^*_i\in\varphi$, lo que es una contradiccion en la
    eleccion de $\sigma_i'$, lo que nos confirma entonces que $\varphi = \emptyset$, y por lo tanto
    $\sigma^w\in\Gamma$.
\end{flushleft}

\section{Segundo grupo de ejercicios para el Parcial \#2}
% --------------------
\begin{flushleft}
    
\end{flushleft}

\section{Dispersión de precios}
% --------------------
\begin{flushleft}
    
\end{flushleft}

\section{NASH EQUILIBRIUM AND THE HISTORY OF ECONOMIC THEORY - Roger B. Myerson}
% --------------------
\begin{flushleft}
    
\end{flushleft}

\newpage

% %%%%%%%%%%%%%%%%%%%%%%%%%%%%%%%%%%%%%%%%%%%%%%%%%%%%%%%%%%
% %%%%%%%%%%%%%%%%%%%%%%%%%%%%%%%%%%%%%%%%%%%%%%%%%%%%%%%%%%
% REFERENCES SECTION
% %%%%%%%%%%%%%%%%%%%%%%%%%%%%%%%%%%%%%%%%%%%%%%%%%%%%%%%%%%
% %%%%%%%%%%%%%%%%%%%%%%%%%%%%%%%%%%%%%%%%%%%%%%%%%%%%%%%%%%
\medskip
\nocite{*}
\bibliography{references.bib} 

\newpage

\end{document}