\documentclass[11pt]{article}
\usepackage{UF_FRED_paper_style}
\onehalfspacing
\setlength{\droptitle}{-5em} %% Don't touch

\title{\text{Microeconomia III - $3^a$ Monitoria}
}

\author{Augusto Rico\\% Name author
    \href{mailto:arico@unal.edu.co}{\texttt{arico@unal.edu.co}}
    }

\date{\today}

\begin{document}
\maketitle

% %%%%%%%%%%%%%%%%%%%%%%%%%%%%%%%%%%%%%%%%%%%%%%%%%%%%%%%%%%
% %%%%%%%%%%%%%%%%%%%%%%%%%%%%%%%%%%%%%%%%%%%%%%%%%%%%%%%%%%
% BODY OF THE DOCUMENT
% %%%%%%%%%%%%%%%%%%%%%%%%%%%%%%%%%%%%%%%%%%%%%%%%%%%%%%%%%%
% %%%%%%%%%%%%%%%%%%%%%%%%%%%%%%%%%%%%%%%%%%%%%%%%%%%%%%%%%%

\section{Equilibrios Mixtos}

\subsection{Correspondencia de mejor respuesta}

\begin{flushleft}
    Una \textit{Correspondencia de mejor respuesta} es el conjunto de estrategias mixtas $\sigma_i^*$ que maximiza la utilidad esperada del jugador $i$ ante una estrategia conjunta $\sigma_{-i}$ de los demas jugadores. Es decir, $$ u_i(\sigma_i^*,\sigma_{-i}) \geq u_i(\sigma_i,\sigma_{-i})$$ de esto se hace evidente entonces que un equilibrio de nash estara caracterizado por la estrategia conjunta de todos los jugadores donde sus correspondencias coinciden, lo que significa que para que $\sigma_i^*$ sea un equilibrio de nash se debe tener conjuntamente que $\sigma_i^*=Mr_i(\sigma_{-i}^*)$ y al mismo tiempo $\sigma_{-i}^*=Mr_{-i}(\sigma_{i}^*)$.

    \begin{example}[Ejemplo: \textit{Caza del ciervo}]
        \begin{flushleft}
            Vamos a mostrar lo anterior con el famoso juego vagamente planteado por \citet{Rousseau}, donde los jugadores tienen dos alternativas: cazar un ciervo, para lo cual necesitarán obligatoriamente la cooperación de otro cazador, o cazar una de las tantas liebres, la cual podrán cazar de forma autónoma. Claramente, la recompensa individual por cazar un ciervo será superior a lo obtenido al decidir cazar una liebre. Este juego lo podemos representar entonces en la siguiente bimatriz\footnote{por facilidad la estrategia $Ciervo$ se representara unicamente con $C$ y de igual forma la estrategia $Liebre$ se representara con $L$}:

            \begin{center}    
                    \setlength{\extrarowheight}{0pt}
                    \begin{tabular}{cc|c|c|}
                        & \multicolumn{1}{c}{}\\
                        & \multicolumn{1}{c}{} & \multicolumn{1}{c}{$C$}  & \multicolumn{1}{c}{$L$} \\\cline{3-4}
                        & $C$ & $5,5$ & $0,3$ \\\cline{3-4}
                        & $L$ & $3,0$ & $3,3$ \\\cline{3-4}
                    \end{tabular}
            \end{center}
    
            donde evidenciamos a simple vista, tal como se explico anteriormente, que existen dos equilibrios puros: $(Ciervo,Ciervo)$ y $(Liebre,Liebre)$, y como tal podemos intuir de forma elemental que es posible que exista al menos un equilibrio mixto, por lo que nos dispondremos a buscarlo.\\~\\
    
            sabiendo que el Cazador $1$ juega la estrategia $Ciervo$ con probabilidad $p$ y la estrategia $Liebre$ con probabilidad $1-p$, y dado que este es un juego simetrico sabemos que el Cazador $2$ jugara estas estrategias con probabilidad $q$, por lo que podemos disponernos a realizar entonces los pagos esperados para cada una de las estrategias del jugador 1:
            \begin{align*}
                u_1(C,q)&=q(5)+(1-q)(0)&=5q\\
                u_1(L,q)&=q(3)+(1-q)(3)&=3
            \end{align*}
            Con estas ecuaciones resueltas, se evidencia de manera clara que la mejor respuesta que puede tomar el Cazador $1$ ante cualquier valor asignado por el Cazador $2$ a $q$ es jugar $Ciervo$ siempre que $u_1(C,q)>u_1(L,q)$, es decir, cuando $5q>3$, que es equivalente a $q>3/5$. De manera análoga, se puede afirmar que elegirá jugar $Liebre$ si $q<3/5$, y será indiferente entre ambas opciones si $q=3/5$, lo que se puede representar en la siguiente correspondencia de mejor respuesta:
            \begin{center}
                $Mr_1(q)=$
                \begin{cases}
                    p=0 &\text{Si  }~~~~~q<3/5\\
                    p\in[0,1] &\text{Si  }~~~~~q=3/5\\
                    p=1 &\text{Si  }~~~~~q>3/5
                \end{cases}
            \end{center}
            Correspondencia que se puede representar de la siguiente forma:
            \begin{center}
                \begin{tikzpicture}
                    \begin{axis}[
                        xlabel={$q$},
                        ylabel={$p$},
                        xmin=0, xmax=1.1,
                        ymin=0, ymax=1.1,
                        xtick={0,3/5,1},
                        ytick={0,1},
                        xticklabels={0, $\frac{3}{5}$, 1},
                        yticklabels={0, 1},
                        axis lines=middle,
                        enlargelimits=true,
                        clip=false,
                        width=8cm,
                        height=6cm,
                        every x tick/.style={opacity=0},
                        every y tick/.style={opacity=0}
                        ]
                        % Función Mr_1(q)
                        \addplot [domain=0:3/5,red,thick] {0};
                        \addplot [domain=3/5:1,red,thick] {1};
                        \addplot [domain=0:1,red,thick] coordinates {(3/5, 0) (3/5, 1)};
                    \end{axis}
                \end{tikzpicture}
            \end{center}
            dado que el juego es simetrico, podemos notar que la correspondencia de mejor respuesta del Cazador $2$ sera:
            \begin{center}
                $Mr_2(p)=$
                \begin{cases}
                    q=0 &\text{Si  }~~~~~p<3/5\\
                    q\in[0,1] &\text{Si  }~~~~~p=3/5\\
                    q=1 &\text{Si  }~~~~~p>3/5
                \end{cases}
            \end{center}
            que se puede representar como:
            \begin{center}
                \begin{tikzpicture}
                    \begin{axis}[
                        xlabel={$q$},
                        ylabel={$p$},
                        xmin=0, xmax=1.1,
                        ymin=0, ymax=1.1,
                        xtick={0,1},
                        ytick={0,3/5,1},
                        xticklabels={0, 1},
                        yticklabels={0, $\frac{3}{5}$, 1},
                        axis lines=middle,
                        enlargelimits=true,
                        clip=false,
                        width=8cm,
                        height=6cm,
                        every x tick/.style={opacity=0},
                        every y tick/.style={opacity=0}
                        ]

                        % Función Mr_2(p)
                        \addplot [domain=0:1,blue,thick] {3/5};
                        \addplot [domain=0:1,blue,thick] coordinates {(0, 0) (0, 3/5)};
                        \addplot [domain=0:1,blue,thick] coordinates {(1, 3/5) (1, 1)};
                    \end{axis}
                \end{tikzpicture}
            \end{center}
            colocando ambas correspondencias en una unica grafica:
            \begin{center}
                \begin{tikzpicture}
                    \begin{axis}[
                        xlabel={$q$},
                        ylabel={$p$},
                        xmin=0, xmax=1.1,
                        ymin=0, ymax=1.1,
                        xtick={0,3/5,1},
                        ytick={0,3/5,1},
                        xticklabels={0, $\frac{3}{5}$, 1},
                        yticklabels={0, $\frac{3}{5}$, 1},
                        axis lines=middle,
                        enlargelimits=true,
                        clip=false,
                        width=8cm,
                        height=6cm,
                        every x tick/.style={opacity=0},
                        every y tick/.style={opacity=0}
                        ]

                        % Función Mr_1(q)
                        \addplot [domain=0:3/5,red,thick] {0};
                        \addplot [domain=3/5:1,red,thick] {1};
                        \addplot [domain=0:1,red,thick] coordinates {(3/5, 0) (3/5, 1)};
                        
                        % Función Mr_2(p)
                        \addplot [domain=0:1,blue,thick] {3/5};
                        \addplot [domain=0:1,blue,thick] coordinates {(0, 0) (0, 3/5)};
                        \addplot [domain=0:1,blue,thick] coordinates {(1, 3/5) (1, 1)};

                        % equilibrios
                        % puros
                        \node[anchor=north] (source) at (axis cs:-0.1,-0.1){\footnotesize \begin{tabular}{c}Eq de Nash\\ Puro $(L,L)$ \end{tabular}};
                        \node (destination) at (axis cs:0,0){};
                        \draw[->](source)--(destination);
                        \addplot[mark=*] coordinates {(0, 0)};

                        \node[anchor=north] (source) at (axis cs:1.35,1.1){\footnotesize \begin{tabular}{c}Eq de Nash\\ Puro $(C,C)$ \end{tabular}};
                        \node (destination) at (axis cs:1,1){};
                        \draw[->](source)--(destination);
                        \addplot[mark=*] coordinates {(1, 1)};

                        \node[anchor=north] (source) at (axis cs:0.3,1.35){\footnotesize \begin{tabular}{c}Eq de Nash\\Mixto \\$(3/5C+2/5L,$\\$3/5C+2/5L)$ \end{tabular}};
                        \node (destination) at (axis cs:3/5,3/5){};
                        \draw[->](source)--(destination);
                        \addplot[mark=*] coordinates {(3/5, 3/5)};
                    \end{axis}
                \end{tikzpicture}
            \end{center}
         \end{flushleft}
    \end{example}
\end{flushleft}

% %%%%%%%%%%%%%%%%%%%%%%%%%%%%%%%%%%%%%%%%%%%%%%%%%%%%%%%%%%
% %%%%%%%%%%%%%%%%%%%%%%%%%%%%%%%%%%%%%%%%%%%%%%%%%%%%%%%%%%
% REFERENCES SECTION
% %%%%%%%%%%%%%%%%%%%%%%%%%%%%%%%%%%%%%%%%%%%%%%%%%%%%%%%%%%
% %%%%%%%%%%%%%%%%%%%%%%%%%%%%%%%%%%%%%%%%%%%%%%%%%%%%%%%%%%
\newpage
\medskip

\nocite{*}
\bibliography{references.bib} 

\newpage

\end{document}