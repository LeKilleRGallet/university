\documentclass[11pt]{article}
\usepackage{UF_FRED_paper_style}
\onehalfspacing
\setlength{\droptitle}{-5em} %% Don't touch

\title{\text{Microeconomia III - $2^a$ Monitoria}
}

\author{Augusto Rico\\% Name author
    \href{mailto:arico@unal.edu.co}{\texttt{arico@unal.edu.co}}
    }

\date{\today}

\begin{document}
\maketitle

% %%%%%%%%%%%%%%%%%%%%%%%%%%%%%%%%%%%%%%%%%%%%%%%%%%%%%%%%%%
% %%%%%%%%%%%%%%%%%%%%%%%%%%%%%%%%%%%%%%%%%%%%%%%%%%%%%%%%%%
% BODY OF THE DOCUMENT
% %%%%%%%%%%%%%%%%%%%%%%%%%%%%%%%%%%%%%%%%%%%%%%%%%%%%%%%%%%
% %%%%%%%%%%%%%%%%%%%%%%%%%%%%%%%%%%%%%%%%%%%%%%%%%%%%%%%%%%

\section{Fallas de Mercado}

\subsection{Monopolio}

\subsubsection{El Problema del Monopolista}

\begin{flushleft}
    Recordemos en la monitoria pasada que el problema que buscaba solucionar el productor era:

    $$\boldsymbol{\max ~ \Pi} = py-c(y)$$
    
    Sin embargo, cuando este productor opera como un monopolista, el problema al que buscará dar solución es diferente. Esto se debe a que, al ser un monopolista, no se asume la existencia de un mercado competitivo. Por el contrario, este productor tiene la capacidad de establecer el precio deseado para el producto vendido. No obstante, es importante tener en cuenta que, dado que este monopolista puede no tener control sobre la fijación de los costos de los insumos, la función de costos sigue siendo invariable. Esto nos lleva a concluir que el problema fundamental al que se enfrenta el monopolista es el siguiente:
    
    $$\boldsymbol{\max ~ \Pi} = p(y)y-c(y)$$
    
    Como se puede observar, en la situación actual el monopolista tiene la capacidad de determinar las cantidades en el mercado, lo que le permite influir en el precio. Por lo tanto, el objetivo del monopolista será determinar las cantidades que maximicen sus ganancias. Dado que la función de costo, como se explicó en la sesión anterior, es convexa $c(\cdot)$, y considerando que la curva de ingreso marginal presenta pendiente negativa, esto garantiza que sea posible derivar con respecto a $y$ para obtener una solución.
    
    $$\frac{\partial \Pi}{\partial y} = \overbrace{p(y)+p'(y)y}^{\text{Ingreso Marginal}}=\underbrace{c'(y)}_{\text{C.Marginal}}$$

\end{flushleft}

\subsubsection{Discriminacion de Precios}

\begin{itemize}
    \item Discriminacion de primer grado (Discriminación perfecta):\\
    Un monopolista practicante de discriminación de primer grado se caracteriza por cobrar a cada comprador el precio máximo que está dispuesto a pagar por unidad. En este caso, el monopolista logra una extracción completa del excedente, lo que implica que $EC=0$.
    \item Discriminacion de segundo grado(De acuerdo a cantidades):\\
    Un monopolista practicante de discriminacion de segundo grado se caracteriza por cobrar dependiendo las cantidades compradas, como puede ser menores precios por compras al por mayor
    \item Discriminacion de tercer grado(De acuerdo a eslasticidades):\\
    Un monopolista practicante de discriminacion de tercer grado se caracteriza por aplicar distintos precios a cada uno de los tipos de compradores, en este caso el monopolista tendra que resolver la siguiente ecuacion:

    $$\max~P^T_yy-c(y)$$

    \begin{example}
    \begin{flushleft}
        asumamos una economia donde:
        \begin{align*}
            p_1 = 2-3y_1,~~p_2=1-2y_2\\
            C(Y)=Y^2|Y=y_1+y_2
        \end{align*}
        por lo que nuestra funcion a maximizar sera:
        $$\max_{y_1,y_2}~(2-3y_1)y_1+(1-2y_2)y_2-(y_1+y_2)^2$$
        derivando:
        \begin{align*}
            \Pi_{y_1}^\partial &= 2-6y_1-2y_1-2y_2= 0 \to& 1-4y_1 = y_2\\
            \Pi_{y_2}^\partial &= 1-4y_2-2y_2-2y_1=0 \to& 1-6y_2-2y_1=0
        \end{align*}
        remplazando $y_2$ en la segunda ecuacion:
        $$1-6(1-4y_1)-2y_1=0$$
        resolvemos para obtener $y_1$:
        \begin{align*}
            0&=\overbrace{1-6}^{-5}\underbrace{+24y_1-2y_1}_{22y_1}\\
            5&=22y_1\\
            y_1&=\frac{5}{22}
        \end{align*}
        remplazamos el valor obtenido de $y_1$ para obtener $y_2$
        \begin{align*}
            y_2&=1-4\frac{5}{22}\\
            y_2&=\frac{1}{11}
        \end{align*}
        remplazando los valores obtenidos en sus respectivas demandas:
        \begin{align*}
            p_1&=2-3(5/22) &\to \frac{29}{22} \approx 1.31\\
            p_2&=1-2(1/11) &\to \frac{9}{11} \approx 0.82
        \end{align*}
        notamos claramente que $p_1>p_2$ esto se debe ya que $\varepsilon_1>\varepsilon_2$\footnote{tenga en cuenta que las elasticidades son valores negativos, por lo que por ejemplo $-1>-2$}
        $$-\frac{\partial q_1}{\partial p_1}\frac{p_1}{q_1}=-\frac{29}{15}$$
        $$-\frac{\partial q_2}{\partial p_2}\frac{p_2}{q_2}=-\frac{9}{2}$$
        sabiendo esto veamos que pasaria entonces si el monopolista no pudiera discriminar:
        $$Y=y_1+y_2 = \frac{p_1-2}{-3}+\frac{p_2-1}{-2}=\frac{-5P+7}{6}$$
        \begin{align*}
            \max_P~&\Pi=P\left(\frac{-5P+7}{6}\right)-\left(\frac{-5P+7}{6}\right)^2\\
            \Pi^\partial_P & = \frac{-10P+7}{6}+\frac{5(-5p+7)}{18}=0\\
            0&=\frac{-55p+56}{18}\\
            P&=\frac{56}{55} ~~~~ Y = \frac{7}{22}
        \end{align*}
        calculamos los beneficios discriminando y sin discriminar:
        \begin{align*}
            \Pi^d &= \frac{29}{22}\frac{5}{22}+\frac{9}{11}\frac{1}{11}-\left(\frac{5}{22}+\frac{1}{11}\right)^2&=\frac{3}{11}\\
            \Pi^m &= \frac{56}{55}\frac{7}{22}-\left(\frac{7}{22}\right)^2&=\frac{49}{220}
        \end{align*}
        notamos claramente que cuando el monopolista discrimina su beneficio es superior
    \end{flushleft}
    \end{example}
    
    
    
\end{itemize}

\begin{flushleft}
    
\end{flushleft}

% %%%%%%%%%%%%%%%%%%%%%%%%%%%%%%%%%%%%%%%%%%%%%%%%%%%%%%%%%%

\subsection{Duopolio de Cournot}
\begin{flushleft}
    El modelo del Duopolio de Cournot presenta un mercado donde existen dos empresas que producen un mismo bien homogeneo a un costo marginal costante $c_i>0$ y enfrentan una demanda conocida por ambas de la forma $p=a-(y_1+y_2)$ entonces la funcion de beneficios de cada firma sera $\Pi_i = (a-(y_1+y_2))y_i-cy_i$ como hemos hecho anteriormente en monopolio esta funcion se puede derivar

    $$\Pi^\partial_{y_i} = a-2y_i-y_{-i}-c=0$$
    
    despejando obtenemos entonces que:
    
    $$y_i=\frac{a-c-y_{-i}}{2}~~~~~~(\text{Curva de Mejor respuesta de }y_i)$$
    
    como notamos a diferencia de casos anteriores, aqui la maximizacion de beneficios depende tambien de cuantas unidades colocara la otra empresa, por lo que en este caso cada empresa va a considerar tambien la mejor respuesta de la otra empresa en sus calculos, por lo que obtendremos entonces que:

    $$y_i=\frac{a-c-\left( \frac{a-c-y_{i}}{2} \right)}{2}$$

    simplificando obtenemos que:

    $$y_i=\frac{a-c}{3}$$

    remplazando en el precio:

    $$p=a-2\left(\frac{a-c}{3}\right)=\frac{a+2c}{3}$$

    tal como se nota $p>c$ por lo que estamos en una falla de mercado.
\end{flushleft}


% %%%%%%%%%%%%%%%%%%%%%%%%%%%%%%%%%%%%%%%%%%%%%%%%%%%%%%%%%%
\section{Juegos de Suma Cero}

ahora pasaremos a resolver ejercicios de suma cero hasta que nos aburramos \href{https://colab.research.google.com/drive/1a9GyRYlV3CaM9UwgZKaUVhhOs_YfPudB?usp=sharing}{aquí}


% %%%%%%%%%%%%%%%%%%%%%%%%%%%%%%%%%%%%%%%%%%%%%%%%%%%%%%%%%%
% %%%%%%%%%%%%%%%%%%%%%%%%%%%%%%%%%%%%%%%%%%%%%%%%%%%%%%%%%%
% REFERENCES SECTION
% %%%%%%%%%%%%%%%%%%%%%%%%%%%%%%%%%%%%%%%%%%%%%%%%%%%%%%%%%%
% %%%%%%%%%%%%%%%%%%%%%%%%%%%%%%%%%%%%%%%%%%%%%%%%%%%%%%%%%%
\newpage
\medskip

\nocite{*}
\bibliography{references.bib} 

\newpage

\end{document}