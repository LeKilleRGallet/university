\documentclass[11pt]{article}
\usepackage{UF_FRED_paper_style}
\onehalfspacing
\setlength{\droptitle}{-5em} %% Don't touch

\title{\text{Microeconomia III - $1^a$ Monitoria}
}

\author{Augusto Rico\\% Name author
    \href{mailto:arico@unal.edu.co}{\texttt{arico@unal.edu.co}}
    }

\date{\today}

\begin{document}
\maketitle

% %%%%%%%%%%%%%%%%%%%%%%%%%%%%%%%%%%%%%%%%%%%%%%%%%%%%%%%%%%
% %%%%%%%%%%%%%%%%%%%%%%%%%%%%%%%%%%%%%%%%%%%%%%%%%%%%%%%%%%
% BODY OF THE DOCUMENT
% %%%%%%%%%%%%%%%%%%%%%%%%%%%%%%%%%%%%%%%%%%%%%%%%%%%%%%%%%%
% %%%%%%%%%%%%%%%%%%%%%%%%%%%%%%%%%%%%%%%%%%%%%%%%%%%%%%%%%%

\section{Teoria del Consumidor}
\subsection{función de utilidad}

\begin{flushleft}
    Uno de los fundamentos clave de la teoría neoclásica es la teoría del consumidor. En esta teoría se parte del supuesto de que todos los consumidores son lo suficientemente racionales como para tener la capacidad de establecer una función de asignación $f_a: X \to \mathbb{R}$, donde $|X| = n \in \mathbb{N} \setminus {0}$, considerando la fecha y el lugar de todas las mercancías que forman parte de la colección $X$.\\
    Habiéndose asignado un valor ordinal a cada una de las mercancías, el consumidor puede generar una relación \textit{n}-aria denotada como $\mathbb{R}^n$\footnote{para este curso asumiremos que esta relacion es siempre positiva (incl. 0), aunque esto no es obligatorio}. Esta relación contiene un conjunto que engloba todas las posibles combinaciones de mercancías, y será denominada como el \textit{conjunto de consumo}, siendo cada uno de sus elementos una \textit{cesta de consumo}. Una forma específica de esta relación es la función de utilidad $f_u:\mathbb{R}_+^n\to\mathbb{R_+}$, la cual deseamos que genere un conjunto convexo $\mathbf{C}$.\\
    Para que este conjunto sea convexo, debe cumplirse que para cualesquiera par de puntos $x_i,x_j\in\mathbf{C}$ con $0\leq\theta\leq1$ se satisfaga que $\theta x_i+(1-\theta)x_j \in \mathbf{C}$. Hecho que nos asegurara que en caso de encontrar un maximo o minimo este sera un punto extremo.
    
    No obstante, existen otros tipos de funciones que también pueden ser útiles, conocidas como funciones cuasicóncavas. Estas se definen como funciones que cumplen $f(\theta x_i+(1-\theta)x_j \leq \min\{f(x_i),f(x_j)\}$, lo que nos permite seguir encontrando soluciones extremas.
    Es importante notar que una función concava también es una función cuasicóncava.

\end{flushleft}

\subsection{restricción presupuestaria}

\begin{flushleft}
    Dado que los recursos son finitos y escasos, los agentes enfrentan restricciones al elegir cestas de consumo. Cada agente tiene un presupuesto ($M$) para consumir bienes $x_i$ a precios  $p_i$, reflejado en la ecuacion $p^Tx\leq M$, limitando su consumo máximo.
\end{flushleft}


\subsection{Problema del consumidor}

\begin{flushleft}
    A partir de esto, se evidencia que el consumidor enfrenta un desafío: encontrar la \textit{cesta de consumo} que, dentro de su presupuesto, maximice su utilidad\footnote{esta cesta en particular se conoce como \textit{cesta optima}}. En otras palabras, debe resolver el problema:
    \begin{center}
        \begin{align*}
            & \boldsymbol{\max ~ U \left(x\right)}\\
            \text{S.A.} \\ &~ p^Tx\leq M
        \end{align*}
    \end{center}

    De manera interesante, podemos notar que el consumidor, sin considerar las elecciones de otros, puede determinar su consumo de cada bien según los precios en la economía. Esto se conoce como demandas marshallianas. Si tenemos la fortuna de que la función sea al menos cuasiconcava estricta y diferenciable con continuidad, podemos aplicar el método de los multiplicadores de Lagrange de la siguiente forma:\\

    \begin{enumerate}
        \item Escribimos el Lagrangiano de nuestro problema de maximizacion \vspace{-0.4cm}
        $$\mathcal{L}=U(x)-\lambda(p^Tx-M)$$\vspace{-1.2cm}
        \item derivamos el lagrangiano obtenido respecto a cada una de las $x$ y las restricciones $\lambda$\\~\\

        \begin{minipage}{0.45\textwidth}
            \begin{align*}
                \frac{\partial\mathcal{L}}{\partial x_1} & = \frac{\partial U(x)}{\partial x_1} -\frac{\partial \lambda(p^Tx-M)}{\partial x_1} = 0 \\~\\
                \frac{\partial\mathcal{L}}{\partial x_2} & = \frac{\partial U(x)}{\partial x_2} -\frac{\partial \lambda(p^Tx-M)}{\partial x_2} = 0 \\
                & \vdots \qquad \qquad \qquad \vdots\\
                \frac{\partial\mathcal{L}}{\partial x_n} & = \frac{\partial U(x)}{\partial x_n} -\frac{\partial \lambda(p^Tx-M)}{\partial x_n} = 0\\~\\
                \frac{\partial\mathcal{L}}{\partial \lambda} & = \frac{\partial U(x)}{\partial \lambda} -\frac{\partial \lambda(p^Tx-M)}{\partial \lambda} = 0 \\
            \end{align*}
        \end{minipage}
        \vrule
        \hspace{0.02\textwidth}
        \begin{minipage}{0.45\textwidth}
            \begin{flushleft}
                Y como sabemos que $\lambda(p^Tx-M)$ es un hiperplano entonces sabemos que su derivada respecto a un componente cualquiera $x_i$ será $\lambda p_i$, obteniendo entonces que:
            \end{flushleft}
            \begin{align*}
                \mathcal{L}_1 & = \frac{\partial U(x)}{\partial x_1} = \lambda p_1 \\
                \mathcal{L}_2 & = \frac{\partial U(x)}{\partial x_2} = \lambda p_2\\
                &  \qquad \vdots \\
                \mathcal{L}_n & = \frac{\partial U(x)}{\partial x_n} = \lambda p_n\\
                \mathcal{L}_\lambda & = p^Tx-M = 0\\
            \end{align*}
        \end{minipage}

        \item habiendo derivado y simplificado todas las ecuaciones podemos entonces elegir cualesquieras ecuaciones $\mathcal{L}_i,\mathcal{L}_j$ exeptuando $\mathcal{L}_\lambda$ y dividirlas de la forma $\mathcal{L}_i/\mathcal{L}_j$ obteniendo entonces:\\

        $$\textbf{TMS}\left\{\frac{\frac{\partial U(x)}{\partial x_i}}{\frac{\partial U(x)}{\partial x_j}}\right.=\left. \frac{p_i}{p_j}\right\}\text{Relacion de Precios}$$

        esta ecuacion se conoce como la \textit{ecuacion de jevons} que relaciona la \textit{Tasa Marginal de Sustitucion} con la \textit{relacion de precios} en otras palabras relaciona la valoracion privada de las mercancias con las valoraciones sociales

        \item con la ecuacion de jevons podemos calcular las demandas marshalianas. tomaremos para el ejemplo la funcion $u(x)=ln(x_1x_2)$ que cumple todos los supuestos necesarios.
        \begin{align*}
            \frac{\frac{\partial U(x)}{\partial x_1}}{\frac{\partial U(x)}{\partial x_2}} = \frac{1/x_1}{1/x_2}&\to  \frac{x_2}{x_1} = \frac{p_1}{p_2} \\
            \text{Despejamos} &: \\
            x_2 &= \frac{p_1x_1}{p_2} && \\
            \text{Remplazamos} &: \\
            p_1x_1 + p_2\left(\frac{p_1x_1}{p_2}\right) &= M && \\
            \text{Despejamos} &: \\
            x_1^m &= \frac{M}{2p_1} && \quad \text{(Demanda Marshaliana de } x_1)\\
            \text{remplazamos} &: \\
            x_2 &= \frac{p_1\frac{M}{2p_1}}{p_2} && \\
            \text{simplificamos} &: \\
            x_2^m &= \frac{M}{2p_2} && \quad \text{(Demanda Marshaliana de } x_2)\\
        \end{align*}
    \end{enumerate}
    
\end{flushleft}

\section{Teoria del Productor}

\subsection{Función de producción}
\begin{flushleft}
    Dentro de la teoría neoclásica, existen tres posibles tipos de funciones de producción: funciones con rendimientos crecientes, constantes y decrecientes.

    \begin{align*}
        &f(tx) > tf(x) \quad (\text{rendimientos crecientes})\\
        &f(tx) = tf(x) \quad (\text{rendimientos constantes})\\
        &f(tx) < tf(x) \quad (\text{rendimientos decrecientes})
    \end{align*}

    En nuestro estudio de mercados de competencia perfecta, nos enfocaremos únicamente en las funciones decrecientes, ya que las empresas con rendimientos constantes y crecientes podrían no tener una solución única en sus problemas de optimización, como veremos más adelante.

    \subsubsection{Funciones de rendimientos decrecientes}
    Para identificar este tipo de funciones, necesitamos que cumplan tres condiciones básicas:
    
    \begin{enumerate}
        \item $f(0) = 0$, de tal forma que no sea posible una producción por generación espontánea.
    
        \item Existencia de marginalidad creciente para todos los insumos ($f_{x_i}'(x) > 0$), lo que implica que a mayor cantidad de insumos corresponda una mayor producción.
    
        \item Existencia de rendimientos marginales decrecientes para todos los insumos ($f_{x_i}''(x) < 0$), lo que implica que a medida que se utilizan cada vez más insumos, la tasa de crecimiento del producto disminuye.\footnote{matriz hessiana definida negativa}
    \end{enumerate}
\end{flushleft}

\subsection{funcion de beneficio}
\begin{flushleft}
    como bien es sabido el beneficio no es mas que la diferencia entre los ingresos y los costos, por lo que nuestra funcion de beneficios sera $\Pi = py-c(x)$
\end{flushleft}

\subsection{Problema del productor}

\begin{flushleft}
    Con base en todo lo expuesto anteriormente, podemos deducir que el problema que el productor busca resolver es el siguiente:

    \begin{center}
        \begin{align*}
            & \boldsymbol{\max ~ \Pi} = py-c(x)\\
            \text{S.A.} \\ &~ y = f(x)
        \end{align*}
    \end{center}
    Este problema puede resolverse si asumimos que la función de producción exhibe rendimientos decrecientes, como se explicó previamente, y si la función de costos es convexa. En tal caso, podemos llegar a la siguiente relación:
    $$F_{x_i}'(x^*) = \frac{c_{x_i}'(x^*)}{p}$$
    Esta relación nos muestra que para que el productor maximice su productividad, la productividad marginal debe igualar al precio real del producto.
\end{flushleft}

% %%%%%%%%%%%%%%%%%%%%%%%%%%%%%%%%%%%%%%%%%%%%%%%%%%%%%%%%%%
% %%%%%%%%%%%%%%%%%%%%%%%%%%%%%%%%%%%%%%%%%%%%%%%%%%%%%%%%%%
% REFERENCES SECTION
% %%%%%%%%%%%%%%%%%%%%%%%%%%%%%%%%%%%%%%%%%%%%%%%%%%%%%%%%%%
% %%%%%%%%%%%%%%%%%%%%%%%%%%%%%%%%%%%%%%%%%%%%%%%%%%%%%%%%%%
\newpage
\medskip

\nocite{*}
\bibliography{references.bib} 

\newpage

\end{document}