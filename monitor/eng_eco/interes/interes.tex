\documentclass{article}
\usepackage{amsmath}

\title{Time value of money}
\author{Augusto Rico}
\date{\today}

\begin{document}
\maketitle
\section{Interes}

\begin{flushleft}
El interes es la base del estudio del valor del dinero en el tiempo.\\
El interes se puede entender simplemente como la variacion de un valor x entre un periodo a y otro periodo b.
~\\
~\\
E.g.:
si se tiene que juan le pide a carlos 100\$ y le promete que transcurrido un tiempo le devolvera 120\$,
entonces el interes es de 20\$.\\

$\text{interes} = \underbrace{120\$}_{\text{dinero momento b}}-\underbrace{100\$}_{\text{dinero momento a}}= \underbrace{20\$}_{\text{interes}}$

~\\
~\\

no obstante para mayor facilidad el interes se calcula como un porcentaje y se le llama tasa de interes por lo que la formula seria
\\~\\
$ \text{interes(\%)}=\frac{120\$-100\$}{100\$}\times 100\%=20\%$\\

~\\
para carlos este interes se pasaria a llamar Tasa Interna de Retorno que representa el interes que gana por prestar el dinero a juan\\
para juan este interes representa el dinero que le paga a carlos por prestarle el dinero.
\end{flushleft}

\section{Inflacion}
\begin{flushleft}

dentro de los estudios economicos existe un concepto llamado la inflacion, que se puede entender 
como una tasa de interes de la economia general. para el curso no es necesario explicar a cabalidad que es la inflacion 
no obstante puede ayudar a comprender mejor el concepto de valor del dinero en el tiempo.\\~\\

la inflacion se entiende como ese aumento generalizado y sostenido de los precios cada año, por lo que cuando los precios suben se necesita mayor cantidad de dinero para poder seguir comprando los mismos bienes algo que nos facilita para poder explicar el valor del dinero en el tiempo\\begin{align*}

E.g.: si el precio de un kilo de papa en el año 2020 costaba 400\$ y en el año 2021 costaba 650\$ si utilizamos la formula de interes tenemos que
\\~\\
$\text{inflacion del kg de papa(\%)}=\frac{650\$-400\$}{400\$}\times 100\%=62.5\%$\\
~\\
por lo que tenemos que la variacion del kilo de papa entre el año 2020 y el año 2021 fue de 62.5\%. Esto significa que mientras en el año 2020 unicamente eran necesarios 400\$ para poder comprar un kilo de papa en el 2021 es necesarios un 62.5\% mas de dinero para poder comprar un kilo de papa.\\~\\

\end{flushleft}

\section{Valor del dinero en el tiempo}
\begin{flushleft}

la definicion anterior de inflacion nos debe ayudar a comprender que no es lo mismo haber recibido 400\$ en el año 2020 que en el año 2021, ya que mientras en el año 2020 este dinero era suficiente para poder comprar un kilo de papa en el año 2021 no es suficiente y exactamente esta es la definicion del valor del dinero en el tiempo, explicar que en dos momentos de tiempo distintos la misma cantidad de dinero no es equivalente\\~\\
siguiendo con el ejemplo de la papa, para un campesino como vendedor de papa no seria beneficioso vender su papa con valor de 400\$ en el 2020 y que se le pagara esta papa en el año 2021 a los mismos 400\$ ya que el campesino estaria perdiendo dinero por la variacion de la inflacion.
el campesino en cambio podria ser indiferente entre aceptar los 400\$ en el 2020 o los 650\$ en el año 2021, y no perderia dinero por la variacion de la inflacion.\\~\\

esta equivalencia entre los 400\$ en el año 2020 y los 650\$ en el año 2021 es lo que se conoce como valor del dinero en el tiempo.

\end{flushleft}

\end{document}